\documentclass{memoir}
\begin{document}
%<*"dx3">
(3)
%</"dx3">
%<*"dx40">
(40)
%</"dx40">
%<*"dx12">
(12)
%</"dx12">
%<*"d0001">
(AFTER THE UNFOLDING OF THE ANTIMENSION)
%</"d0001">
%<*"d0002">
(Aloud)
%</"d0002">
%<*"d0003">
(Aloud) For yours it is to have mercy and to save us, O our God, and to you we give glory, to the Father, the Son and the Holy Spirit, now and for ever, and to the ages of ages.
%</"d0003">
%<*"d0004">
(BEFORE THE UNFOLDING OF THE ANTIMENSION)
%</"d0004">
%<*"d0005">
(On Sundays: he who rose from the dead) Christ, our true God, through the prayers of his most pure and holy Mother, of our Father among the Saints John Chrysostom, Archbishop of Constantinople (Or: Basil the Great, Archbishop of Caesarea in Cappadocia), and of all the Saints, have mercy on us and save us, for he is good and loves mankind.
%</"d0005">
%<*"d0006">
(The Dismissal)
%</"d0006">
%<*"d0007">
(The English text of this hymn is missing, because it was either inaccessible at the time of publication or unavailable due to copyright restrictions.)
%</"d0007">
%<*"d0008">
(The Prokeimenon)
%</"d0008">
%<*"d0009">
(three times).
%</"d0009">
%<*"d0010">
*******************************************
%</"d0010">
%<*"d0011">
*Deacon, coming near the Holy Doors and facing the People: and to the ages of ages.
%</"d0011">
%<*"d0012">
*Deacon: Help us, save us, have mercy on us, and keep us, O God, by your grace.
%</"d0012">
%<*"d0013">
*Deacon: Let us attend.
%</"d0013">
%<*"d0014">
*Deacon: Let us pray to the Lord.
%</"d0014">
%<*"d0015">
*Deacon: Wisdom.
%</"d0015">
%<*"d0016">
1. At Vespers
%</"d0016">
%<*"d0017">
1. In honour and memory of the great Captains, the Archangels Michael and Gabriel, and of all the Bodiless Powers of heaven.
%</"d0017">
%<*"d0018">
1. In honour and memory of the honoured and glorious Prophet, Forerunner and Baptist, John.
%</"d0018">
%<*"d0019">
1st Prayer
%</"d0019">
%<*"d0020">
1st Priest: Give the Command!
%</"d0020">
%<*"d0021">
1st Priest: Give the command, O Holy Master, for the one who is presented before you!
%</"d0021">
%<*"d0022">
2. At Matins
%</"d0022">
%<*"d0023">
2. Of the holy, glorious Prophets Moses and Aaron, Elias, Elissaios, David, son of Jesse, the Three Holy Youths, the prophet Daniel and all the holy Prophets.
%</"d0023">
%<*"d0024">
2. Of the honoured and glorious Prophet, Forerunner and Baptist, John; of the holy, glorious Prophets Moses and Aaron, Elias, Elissaios, David, son of Jesse, the Three Holy Youths, the prophet Daniel and all the holy Prophets.
%</"d0024">
%<*"d0025">
2nd Prayer
%</"d0025">
%<*"d0026">
2nd Priest: Command!
%</"d0026">
%<*"d0027">
3. At the end of the Liturgy
%</"d0027">
%<*"d0028">
3. Of the holy, glorious and all-praised Apostles Peter and Paul, the Twelve, the Seventy and all the holy Apostles.
%</"d0028">
%<*"d0029">
3rd Prayer
%</"d0029">
%<*"d0030">
4. Of our Fathers among the Saints, great Hierarchs and Ecumenical Teachers, Basil the Great, Gregory the Theologian and John Chrysostom, Athanasios and Cyril, Nicolas of Myra, and all holy hierarchs.
%</"d0030">
%<*"d0031">
40 times (in groups of 10) and again 3 times.
%</"d0031">
%<*"d0032">
4th Prayer
%</"d0032">
%<*"d0033">
5. Of the holy Protomartyr and Archdeacon Stephen; of the holy glorious Great Martyrs, George, the Victorious, Demetrios Myrovlites, Theodore the Recruit and Theodore the General; of the holy, glorious Protomartyr and Equal of the Apostles, Thekla, of Barbara, Katherine, Marina and Paraskevi; and of all holy martyrs.
%</"d0033">
%<*"d0034">
5th Prayer
%</"d0034">
%<*"d0035">
6. Of our venerable and God-bearing Fathers, Antony the Great, Efthymios, Savvas, Onouphrios, Peter and Athanasios of Athos; of our Venerable and God-bearing Mothers, Pelagia, Theodosia, Euphrosyne, Mary of Egypt; and of all holy ascetics.
%</"d0035">
%<*"d0036">
6th Prayer
%</"d0036">
%<*"d0037">
7. Of the holy Wonderworkers and Unmercenary Physicians, Cosmas and Damian, Cyrus and John, Panteleimon and Hermolaos, and all holy Unmercenary Saints.
%</"d0037">
%<*"d0038">
7th Prayer
%</"d0038">
%<*"d0039">
8. Of the holy and righteous Forebears of God, Joachim and Anne, of N. (the Saint of the day), whose memory we celebrate, and of all the Saints, at whose intercessions visit us, O God.
%</"d0039">
%<*"d0040">
9. Of our Father among the Saints John Chrysostom, Archbishop of Constantinople (or Of our Father among the Saints Basil the Great, Archbishop of Caesarea in Cappadocia).
%</"d0040">
%<*"d0041">
A Christian end to our life, painless, unashamed and peaceful, and a good defence before the dread judgement seat of Christ, let us ask.
%</"d0041">
%<*"d0042">
AND
%</"d0042">
%<*"d0043">
AND GREAT ENTRANCE
%</"d0043">
%<*"d0044">
APOLYTIKIA FOR THE GREAT FEASTS
%</"d0044">
%<*"d0045">
APOLYTIKIA FROM THE TRIODION AND PENTECOSTARION
%</"d0045">
%<*"d0046">
APOLYTIKIA OF THE RESURRECTION
%</"d0046">
%<*"d0047">
APPENDIX 1
%</"d0047">
%<*"d0048">
AT THE GRAVESIDE
%</"d0048">
%<*"d0049">
After the 1st Kathisma
%</"d0049">
%<*"d0050">
After the 2nd Kathisma
%</"d0050">
%<*"d0051">
After the 3rd Ode
%</"d0051">
%<*"d0052">
After the Dismissal the Deacon censes the holy Prothesis. Then he goes and censes the Holy Table all round, crosswise, saying in a low voice:
%</"d0052">
%<*"d0053">
After the completion of the Antiphon he comes and stands in his usual place, bows and says the
%</"d0053">
%<*"d0054">
After the prayer the Priest and the Deacon say the Cherubic Hymn three times, as follows:
%</"d0054">
%<*"d0055">
After this he gives the Dismissal, saying:
%</"d0055">
%<*"d0056">
After this he takes the ninth particle and places it below the eighth, so completing the third rank, saying:
%</"d0056">
%<*"d0057">
After this, taking a seventh particle, he places it beside the fourth, so staring the third rank, saying:
%</"d0057">
%<*"d0058">
Again and again in peace, let us pray to the Lord.
%</"d0058">
%<*"d0059">
Again and again, in peace let us pray to the Lord.
%</"d0059">
%<*"d0060">
Again we pray for all pious and Orthodox Christians.
%</"d0060">
%<*"d0061">
Again we pray for our Archbishop (N) and our Bishop (N), and all our brothers and sisters in Christ.
%</"d0061">
%<*"d0062">
Again we pray for the repose of the soul of the departed servant of God (name), and for the forgiveness of his/her every transgression, voluntary and involuntary.
%</"d0062">
%<*"d0063">
Again we pray for the repose of the soul of the departed servant of God [name], and for the forgiveness of his/her every transgression, voluntary and involuntary.
%</"d0063">
%<*"d0064">
Again we pray for the repose of the soul of the servant of God, N., who has fallen asleep, and that he/she may be pardoned every offence, both voluntary and involuntary.
%</"d0064">
%<*"d0065">
Again we pray mercy, life, peace, health, salvation, visitation and forgiveness of sins for the servants of God here present, and that they may be pardoned every offence both voluntary and involuntary.
%</"d0065">
%<*"d0066">
All you who trod in life the hard and narrow way; all you who took the Cross as a yoke, and followed me in faith, come, enjoy that heavenly rewards and crowns which I have prepared for you.
%</"d0066">
%<*"d0067">
All-Holy Trinity, have mercy on us. Lord be merciful to our sins.
%</"d0067">
%<*"d0068">
All-holy Lady, Mother of God, the light of my darkened soul, my hope, protection, refuge, comfort, joy, I thank you, because you have made me, the unworthy, worthy to become a partaker in the most pure Body and precious Blood of your Son. But, O you who gave birth to the true Light, enlighten the spiritual eyes of my heart; you who bore the source of immortality, give life to me, who have been slain by sin; you the compassionate Mother of the merciful God, have mercy on me and give me compunction and contrition in my heart, and humility in my ideas, and release from the imprisonment of my thoughts. And count me worthy, until my last breath, to receive without condemnation the hallowing of the most pure Mysteries, for healing of soul and body; and grant me tears of repentance and thanksgiving, to praise and glorify you all the days of my life.
%</"d0068">
%<*"d0069">
All-holy Trinity, have mercy on us. Lord, cleanse us from our sins. Master, pardon our iniquities. Holy One, visit and heal our infirmities for your name's sake. Lord, have mercy. Lord, have mercy. Lord, have mercy.
%</"d0069">
%<*"d0070">
All-holy Trinity, have mercy on us. Lord, cleanse us from our sins. Master, pardon our iniquities. Holy One, visit and heal our infirmities for your name's sake.
%</"d0070">
%<*"d0071">
All-holy Trinity, have mercy on us. Lord, cleanse us from our sins. Master, pardon our iniquities. Holy One, visit and heal our infirmities for your name's sake. Lord, have mercy. Lord, have mercy. Lord, have mercy.
%</"d0071">
%<*"d0072">
All-holy Virgin Theotokos, guide the works of our hands, and intercede for the forgiveness of all our faults, as we chant the hymn of the Angels.
%</"d0072">
%<*"d0073">
Alleluia x3
%</"d0073">
%<*"d0074">
Alleluia, Alleluia, Alleluia. (with the verses).
%</"d0074">
%<*"d0075">
Alleluia, Alleluia, Alleluia. Glory to you, O God. (x3)
%</"d0075">
%<*"d0076">
Alleluia, Alleluia, Alleluia. Glory to you, O God. (x3). Metanias (x3).
%</"d0076">
%<*"d0077">
Alleluia, alleluia, alleluia.
%</"d0077">
%<*"d0078">
Alleluia. Alleluia. Alleluia.
%</"d0078">
%<*"d0079">
Alleluia. Alleluia. Alleluia. Glory to you, O God. (three times).
%</"d0079">
%<*"d0080">
Aloud
%</"d0080">
%<*"d0081">
(Aloud.)
%</"d0081">
%<*"d0082">
Also we pray for mercy, life, peace, health, salvation, visitation, pardon and forgiveness of sins for the servants of God, all devout and Orthodox Christians, those who dwell in or visit this city and parish, the wardens and members of this church and their families.
%</"d0082">
%<*"d0083">
Also we pray for mercy, life, peace, health, salvation, visitation, pardon and forgiveness of sins for the servants of God, all devout and Orthodox Christians, those who dwell in or visit this city and parish, the wardens and members of this church and their families; [and for the servants of God N and N (Here he may name those for whom he has been asked to pray), and all who have asked for our prayers, unworthy though we are.]
%</"d0083">
%<*"d0084">
Also we pray for mercy, life, peace, health, salvation, visitation, pardon and forgiveness of sins of the servants of God, the parishioners, members, wardens, benefactors of this holy church, and of his servants (and he commemorates the names of those who have offered the loaves) who celebrate this holy feast, and who offer these gifts.
%</"d0084">
%<*"d0085">
Also we pray for our Archbishop N.
%</"d0085">
%<*"d0086">
Also we pray for our Archbishop N. and all our brotherhood in Christ.
%</"d0086">
%<*"d0087">
Also we pray for our Archbishop N., and for all our brotherhood in Christ.
%</"d0087">
%<*"d0088">
Also we pray for the blessed and ever-remembered founders of this holy church, and for all our brothers and sisters who have gone to their rest before us, and who lie asleep here in the true faith; and for the Orthodox everywhere[, and for the servants of God N and N (Here he may name those for whom he has been asked to pray), and that they may be pardoned all their offences, both voluntary and involuntary].
%</"d0088">
%<*"d0089">
Also we pray for the protection of this holy church, this city and every city, town and village from plague, famine, earthquake, flood, fire, sword, invasion by enemies, civil war and sudden death; and that our good God, who loves mankind, will be merciful, kindly and easily entreated, and turn away and dispel all wrath and disease stirred up against us, and deliver us from his just threat that hangs over us, and have mercy on us.
%</"d0089">
%<*"d0090">
Also we pray for the repose of all devout and Orthodox Christians who have fallen asleep before us in hope of resurrection, Rulers, Patriarchs, Bishops, Priests, Deacons, Monks, Nuns, Parents, Grandparents, Great Grandparents and Ancestors, those from the beginning until the last, and that they may be pardoned every offence, both voluntary and involuntary.
%</"d0090">
%<*"d0091">
Also we pray for the repose of the soul of the servant of God, N., who has fallen asleep, and that he/she may be pardoned every offence, both voluntary and involuntary.
%</"d0091">
%<*"d0092">
Also we pray for the repose of the souls of the servants of God, N. and N., who have fallen asleep, and that they may be pardoned every offence, both voluntary and involuntary.
%</"d0092">
%<*"d0093">
Also we pray for the servants of God who make this supplication, and for the servants of God (and he commemorates the names of those for whom the supplication is being made).
%</"d0093">
%<*"d0094">
Also we pray for those who bring offerings, those who care for the beauty of this holy and venerable house, for those who labour in its service, for those who sing, and for the people here present, who await your great and rich mercy.
%</"d0094">
%<*"d0095">
Also we pray that the Lord, our God, will hear the voice of supplication of us sinners, and have mercy on us.
%</"d0095">
%<*"d0096">
Although it is the practice in many parishes to hold the Artoklasia at the end of the Liturgy, we strongly recommend that it takes place either at the end of Vespers, or at the end of the Doxology before the Liturgy.
%</"d0096">
%<*"d0097">
Always, now and for ever, and to the ages of ages. Amen.
%</"d0097">
%<*"d0098">
Amen
%</"d0098">
%<*"d0099">
Amen, Amen, Amen.
%</"d0099">
%<*"d0100">
Amen.
%</"d0100">
%<*"d0101">
Amen. (And chant the following hymns)
%</"d0101">
%<*"d0102">
Amen. Holy God, Holy Strong, Holy Immortal, have mercy on us (x3).
%</"d0102">
%<*"d0103">
Amen. Holy God, Holy Strong, Holy Immortal, have mercy on us. (three times).
%</"d0103">
%<*"d0103a">
Amen. Holy God, Holy Strong, Holy Immortal, have mercy on us.
%</"d0103a">
%<*"d0104">
Amen. May the Lord God strengthen the holy and pure faith of devout and orthodox Christians, with his holy Church, unto ages of ages.
%</"d0104">
%<*"d0105">
Amen. kisses the Priest's right hand and goes out and stands in his usual place and says the
%</"d0105">
%<*"d0106">
Amen. takes it. Having kissed the Priest's hand, he goes out through the Holy Doors, and preceded by lights makes his way to the Ambo.
%</"d0106">
%<*"d0107">
An angel of peace, a faithful guide, a guardian of our souls and bodies, let us ask of the Lord.
%</"d0107">
%<*"d0108">
And Dismissal.
%</"d0108">
%<*"d0109">
And Psalm 50:
%</"d0109">
%<*"d0110">
And again.
%</"d0110">
%<*"d0111">
And again:
%</"d0111">
%<*"d0112">
And as an unblemished lamb before its shearer is dumb, so he does not open his mouth.
%</"d0112">
%<*"d0113">
And have mercy on me. Alleluia.
%</"d0113">
%<*"d0114">
And he covers the chalice for the moment with one of the covers.
%</"d0114">
%<*"d0115">
And he enters the Sanctuary.
%</"d0115">
%<*"d0116">
And he receives the Holy Bread with fear and great care; and having wiped his hand over the Paten with the sponge he says:
%</"d0116">
%<*"d0117">
And immediately the prayers of thanksgiving.
%</"d0117">
%<*"d0118">
And of all our fellow Orthodox who have fallen asleep in the hope of resurrection to everlasting life, in communion with you, Lord, Lover of mankind.
%</"d0118">
%<*"d0119">
And taking a fifth particle he places it below the fourth, saying:
%</"d0119">
%<*"d0120">
And taking a fourth particle he places it next to the first, so starting the second rank, saying:
%</"d0120">
%<*"d0121">
And taking a second particle he places it below the first, saying:
%</"d0121">
%<*"d0122">
And taking a sixth particle he places it below the fifth, so completing the second rank, saying:
%</"d0122">
%<*"d0123">
And taking a third particle he places it below the second, thus completing the first rank, saying:
%</"d0123">
%<*"d0124">
And that he would count us worthy to listen to the holy Gospel, let us pray to the Lord God.
%</"d0124">
%<*"d0125">
And the Catechumen, or the Godparent, answers:
%</"d0125">
%<*"d0126">
And the Deacon, having answered
%</"d0126">
%<*"d0127.title.doc">
And the Dismissal
%</"d0127.title.doc">
%<*"d0127.title.toc">
And the Dismissal
%</"d0127.title.toc">
%<*"d0127.title.heading">
And the Dismissal
%</"d0127.title.heading">
%<*"d0128">
And the Liturgy begins.
%</"d0128">
%<*"d0129">
And the Priest hands him the Gospel. The Deacon, saying
%</"d0129">
%<*"d0130">
And the Priest says this Prayer:
%</"d0130">
%<*"d0131">
And the Priest says this prayer:
%</"d0131">
%<*"d0132">
And the Priest, as he enters, says to him:
%</"d0132">
%<*"d0133">
And the Priest, blessing the entrance, says, in a low voice:
%</"d0133">
%<*"d0134">
And the following Troparia:
%</"d0134">
%<*"d0135">
And the person appointed to read the Psalms continues:
%</"d0135">
%<*"d0136">
And the prayer "O God of spirits...," and the rest as above.
%</"d0136">
%<*"d0137">
And they both bow three times before the Offering. Then the Deacon takes the censer and says:
%</"d0137">
%<*"d0138">
And they complete the Cherubic Hymn:
%</"d0138">
%<*"d0139">
And they go to the Throne. As they go the Priest says:
%</"d0139">
%<*"d0140">
And to your Spirit.
%</"d0140">
%<*"d0141">
And to your Spirit. Alleluia, Alleluia, Alleluia. Tone 6.
%</"d0141">
%<*"d0142">
And to your spirit.
%</"d0142">
%<*"d0143">
And unto your spirit.
%</"d0143">
%<*"d0144">
And we begin the Six Psalms, listening with complete silence and compunction. The Superior, or the designated reader, with devotion and fear of God says:
%</"d0144">
%<*"d0145">
And when he has spoken three times to each of them he makes the sign of the Cross with the ring on their foreheads and places the rings on their right fingers.
%</"d0145">
%<*"d0146">
And with your spirit.
%</"d0146">
%<*"d0147">
Angelic Powers were at your grave, and those who guarded it became as dead, and Mary stood by the tomb, seeking your most pure Body. You despoiled Hades and emerged unscathed; you met the Virgin and granted life. Lord, risen from the dead, glory to you!
%</"d0147">
%<*"d0148">
Anonymous.
%</"d0148">
%<*"d0149">
Anonymous. To the Most Holy Mother of God.
%</"d0149">
%<*"d0150">
Another, for Palm Sunday only.
%</"d0150">
%<*"d0151">
Apolytikion of Saint John Chrysostom.
%</"d0151">
%<*"d0152">
Apolytikion of the Dedication. Tone 4.
%</"d0152">
%<*"d0153">
As first-fruits of nature, Lord, creation's Planter, the world offers you, the god-bearing Martyrs; at their intercessions preserve your Church in profound peace, through the Mother of God, O most merciful.
%</"d0153">
%<*"d0154">
As he censes from in front of the holy Table, the Priest says, aloud:
%</"d0154">
%<*"d0155">
As he cuts along the left side, that is beside the letters XC.KA:
%</"d0155">
%<*"d0156">
As he cuts along the lower side:
%</"d0156">
%<*"d0157">
As he cuts along the upper side:
%</"d0157">
%<*"d0158">
As the beauty of the firmament above, O Lord, you have displayed the loveliness of the holy Dwelling of your glory below. Strengthen it to age on age and accept, through the Mother of God, our supplications which are offered in it without ceasing to you, the Life and Resurrection of all.
%</"d0158">
%<*"d0159">
As they kiss that of the Mother of God:
%</"d0159">
%<*"d0160">
As they venerate the icon of the Patron of the Church they say the appropriate troparion.
%</"d0160">
%<*"d0161">
As we celebrate the memory of the Dedication, loving and all-powerful Lord, we glorify you, the giver of sanctification, asking that the senses of our souls may be sanctified at the intercession of the glorious Champions.
%</"d0161">
%<*"d0162">
As you are source of compassion, grant us pity, Mother of God. Look on a people who have sinned, and show your power as always; for hoping in you we cry: Hail, as once Gabriel did, the chief Captain of the Bodiless Powers.
%</"d0162">
%<*"d0163">
As you go to receive Communion say the following verses to yourself:
%</"d0163">
%<*"d0164">
As you were baptized in the Jordan, Lord, the worship of the Trinity was made manifest: for the voice of the Father bore witness to you, naming you the Beloved Son; and the Spirit, in the form of a dove, confirmed the sureness of the word. Christ God, who appeared and enlightened the world, glory to you.
%</"d0164">
%<*"d0165">
Assuring us before your Passion of the general resurrection, you raised Lazarus from the dead, O Christ God: therefore, like the Children, we also carry tokens of victory, and we cry to you, the Conqueror of death: Hosanna in the highest! Blessed is he who comes in the name of the Lord.
%</"d0165">
%<*"d0166">
At Matins. At the 3 Readings from the Psalter, the Kathisma of the Feast, of the Dedication and again of the Feast. The Canons of the Feast and of the Dedication. The Exapostilarion of the Feast, of the Dedication, of the Saint of the Church and that. of the Feast again. At Lauds, 4 Stichera of the Feast, 2 of the Dedication and 2 of the feasted Saint. \u2018Glory' of the Feast. \u2018Both now' of the Dedication. Great Doxology and the Apolytikion of the Feast.
%</"d0166">
%<*"d0167">
At that time, they brought unto Jesus infants, that he would touch them: but when his disciples saw it, they rebuked them. But Jesus called them unto him, and said, Suffer little children to come unto me, and forbid them not: for of such is the kingdom of God. Verily I say unto you, Whosoever shall not receive the kingdom of God as a little child shall in no wise enter therein. And they that heard it said, Who then can be saved? And he said, The things which are impossible with men are possible with God.
%</"d0167">
%<*"d0168">
At the mid-point of the Feast, O Saviour, water my thirsty soul with streams of true devotion; for you cried out to all: Any who thirst, let them come to me, and let them drink! O Source of life, Christ our God, glory to you!
%</"d0168">
%<*"d0169">
At the mid-point of the feast according to the Law, Maker of all things and Master, you said to those who were present, O Christ God, \u2018Come, and draw the water of immortality'. Therefore we fall down before you and with faith we cry: Grant us your mercies, for you are the source of our life.
%</"d0169">
%<*"d0170">
At the prayers of all your Saints and of the Mother of God, grant us your peace, O Lord, and have mercy on us, for you alone are compassionate.
%</"d0170">
%<*"d0171">
At the prayers of the Mother of God, O Saviour, save us.
%</"d0171">
%<*"d0172">
August 15th. The Dormition of the Mother of God, and until the Close on the 23rd (or the 28th).
%</"d0172">
%<*"d0173">
August 6th. The Transfiguration of our Saviour, Jesus Christ, and until the Close on the 13th.
%</"d0173">
%<*"d0174">
BLESSING FOR A CAKE FOR ST. PHANOURIOS
%</"d0174">
%<*"d0175">
BLESSING OF PALMS
%</"d0175">
%<*"d0176">
Because you, O God, are merciful and love mankind, and to you we give glory, Father, Son and Holy Spirit, now and for ever, and to the ages of ages.
%</"d0176">
%<*"d0177">
Before the closing "Christ is risen from the dead..." the Priest pours on the body olive oil saying:
%</"d0177">
%<*"d0178">
Behold, again I draw near to Christ, our immortal King and God.
%</"d0178">
%<*"d0179">
Bishop: Blessed is our God, always now and for ever, and to the ages of ages.
%</"d0179">
%<*"d0180">
Bishop: Glory to you, holy Trinity, our God, to the ages of ages. Amen.
%</"d0180">
%<*"d0181">
Bishop: Peace to all.
%</"d0181">
%<*"d0182">
Bishop: The Lord mighty and powerful, the Lord powerful in war. Lift up your gates you rulers; and be lifted up you eternal gates, and the king of glory will enter.
%</"d0182">
%<*"d0183">
Bless, Master, the holy union.
%</"d0183">
%<*"d0184">
Bless.
%</"d0184">
%<*"d0185">
Blessed are the blameless in the way, who walk in the law of the Lord. Alleluia.
%</"d0185">
%<*"d0186">
Blessed are you, Christ our God, who proclaimed the fishermen to be most wise by sending down to them the Holy Spirit, and so through them catching the whole world in a net. Lover of mankind, glory to you!
%</"d0186">
%<*"d0187">
Blessed are you, Lord, teach me your statutes.
%</"d0187">
%<*"d0188">
Blessed are you, O Lord, * the God of our Fathers, * and praised and glorified is your name * to the ages. Amen.
%</"d0188">
%<*"d0189">
Blessed are you, O Lord, teach me your statutes.
%</"d0189">
%<*"d0190">
Blessed art You, O Lord, teach me Your statutes.
%</"d0190">
%<*"d0191">
Blessed is He Who Is, Christ our true God, always, now and for ever, and to the ages of ages.
%</"d0191">
%<*"d0192">
Blessed is he who comes in the name of the Lord.
%</"d0192">
%<*"d0193">
Blessed is he who comes in the name of the Lord. The Lord is God and has appeared to us.
%</"d0193">
%<*"d0194">
Blessed is our God always, now and ever, and to the ages of ages.
%</"d0194">
%<*"d0195">
Blessed is our God always, now and for ever, and to the ages of ages.
%</"d0195">
%<*"d0196">
Blessed is our God always, now and for ever, and unto the ages of ages.
%</"d0196">
%<*"d0197">
Blessed is our God, always now and for ever, and to the ages of ages.
%</"d0197">
%<*"d0198">
Blessed is our God, always, now and ever, and to the ages of ages.
%</"d0198">
%<*"d0199">
Blessed is our God, always, now and for ever, and to the ages of ages.
%</"d0199">
%<*"d0200">
Blessed is our God, who has been thus well-pleased. Glory to you. (three times).
%</"d0200">
%<*"d0201">
Blessed is our God. Always, now and for ever, and to the ages of ages. Amen.
%</"d0201">
%<*"d0202">
Blessed is our God. and then turns to the People, shows them the Chalice and continues, aloud:
%</"d0202">
%<*"d0203">
Blessed is the Kingdom of the Father, and of the Son, and of the Holy Spirit, now and for ever, and to the ages of ages.
%</"d0203">
%<*"d0204">
Blessed is the union of your holy things, always, now and for ever, and to the ages of ages. Amen.
%</"d0204">
%<*"d0205">
Both now and for ever and to the ages of ages. Amen.
%</"d0205">
%<*"d0206">
Both now and for ever, and to the ages of ages. Amen.
%</"d0206">
%<*"d0207">
Both now and for ever, and to the ages of ages. Amen. Alleluia.
%</"d0207">
%<*"d0208">
Both now and for ever, and to the ages of ages. Amen. Have mercy on your servant.
%</"d0208">
%<*"d0209">
Both now and for ever, and to the ages of ages. Amen. Theotokion.
%</"d0209">
%<*"d0210">
Both now.
%</"d0210">
%<*"d0211">
Both now. Theotokion.
%</"d0211">
%<*"d0212">
Bow your heads to the Lord.
%</"d0212">
%<*"d0213">
Brethren, Christ being raised from the dead dies no more; death has no more dominion over him. For in that he died, he died unto sin once: but in that he lives, he lives unto God. Likewise reckon ye also yourselves to be dead indeed unto sin, but alive unto God through Jesus Christ our Lord.
%</"d0213">
%<*"d0214">
Brethren, I do not wish you to be ignorant about those who have fallen asleep, so that you may not grieve like the rest who have no hope. For if we believe that Jesus died and rose again, so too God will bring with him those who sleep through Jesus. We tell you this by the Lord's word, that we who are left alive at the Lord's coming will by no means precede those sleep. Because the Lord himself will descend from heaven with a shout, at the voice of an Archangel and with the trumpet of God, and the dead in Christ will rise first. Then we who are left alive will be snatched up together with them in the clouds to meet the Lord in the air; and so we shall all be with the Lord.
%</"d0214">
%<*"d0215">
Buried with you through Baptism, Christ our God, we have been granted immortal life by your Resurrection, and we sing your praises, as we cry: Hosanna in the highest! Blessed is he who comes in the name of the Lord.
%</"d0215">
%<*"d0216">
By Saint Basil the Great.
%</"d0216">
%<*"d0217">
By St Cosmas the Melodist.
%</"d0217">
%<*"d0218">
By Symeon Metaphrastes.
%</"d0218">
%<*"d0219">
By rising from the tomb * and bursting through the bonds of Hades, * you abolished the sentence of death, O Lord, * delivering all from the snares of the foe. * Having shown yourself to your Apostles, * you sent them out to preach, * and through them you gave peace * to the inhabited world, * you who alone are full of mercy.
%</"d0219">
%<*"d0220">
By the Holy Spirit every soul is given life, by cleansing it is exalted, it is made bright by the threefold Unity in a sacred mystery.
%</"d0220">
%<*"d0221">
By the coming down of the all-holy Spirit of the Father through the voice of the Archangel you conceived the One who shares the Father's throne and substance, O Mother of God, the recalling of Adam.
%</"d0221">
%<*"d0222">
By the prayers of our holy fathers, Lord Jesus Christ our God, have mercy upon us and save us.
%</"d0222">
%<*"d0223">
CHERUBIC HYMN
%</"d0223">
%<*"d0224">
CHOIR
%</"d0224">
%<*"d0225">
CLERGY
%</"d0225">
%<*"d0226">
CLERGY our Archbishop, N,
%</"d0226">
%<*"d0227">
 I have united myself to him.
%</"d0227">
%<*"d0228">
Choir
%</"d0228">
%<*"d0229">
%</"d0229">
Choir 1:
%<*"d0230">
Choir 2:
%</"d0230">
%<*"d0231">
Christ has risen from the dead, by death he has trampled on death and to those in the graves he has given life.
%</"d0231">
%<*"d0232">
Christ is in our midst. to which the junior answers:
%</"d0232">
%<*"d0233">
Christ is risen from the dead, by death he has overcome death, and to them in the graves has he given life.
%</"d0233">
%<*"d0234">
Christ is risen from the dead, by death he has overcome death, and to them in the graves has he given life.(x2)
%</"d0234">
%<*"d0235">
Christ is risen from the dead...(x3)
%</"d0235">
%<*"d0236">
Christ is risen from the dead...x (3)
%</"d0236">
%<*"d0237">
Christ, the joy of all, the truth, the light, the life, the resurrection of the world, in his goodness has appeared to those on earth, and has himself become the pattern of the resurrection, granting to all divine forgiveness.
%</"d0237">
%<*"d0238">
Clothed as in purple and fine linen with the blood of your Martyrs throughout the world, your Church cries out to you through them, Christ God: Send down your pity on your people; grant peace to your commonwealth, and to our souls your great mercy.
%</"d0238">
%<*"d0239">
Come, let us worship and fall down before Christ himself, the King, our God.
%</"d0239">
%<*"d0240">
Come, let us worship and fall down before Christ the King, our God.
%</"d0240">
%<*"d0241">
Come, let us worship and fall down before Christ. Son of God, risen from the dead, save us who sing to you: Alleluia!
%</"d0241">
%<*"d0242">
Come, let us worship and fall down before the King our God.
%</"d0242">
%<*"d0243">
Come, let us worship and fall down before the King, our God.
%</"d0243">
%<*"d0244">
Commemorating our all\u001Eholy, pure, most blessed and glorious Lady, Mother of God and Ever\u001EVirgin Mary, with all the Saints, let us entrust ourselves and one another and our whole life to Christ our God.
%</"d0244">
%<*"d0245">
Commemorating our all-holy, pure, most blessed and glorious Lady,
%</"d0245">
%<*"d0246">
Commemorating our all-holy, pure, most blessed and glorious Lady, Mother of God and Ever-Virgin Mary, with all the Saints, let us entrust ourselves and one another and our whole life to Christ our God.
%</"d0246">
%<*"d0247">
Cover, Master.
%</"d0247">
%<*"d0248">
Cutting the particle for the Mother of God, he places it to the right of the Lamb, near the middle, saying:
%</"d0248">
%<*"d0249">
DEACON
%</"d0249">
%<*"d0250">
DISTRIBUTION OF HOLY COMMUNION
%</"d0250">
%<*"d0251">
Deacon
%</"d0251">
%<*"d0252">
Deacon, again draw near.
%</"d0252">
%<*"d0253">
Deacon, bowing his head, says to the Priest:
%</"d0253">
%<*"d0254">
Deacon, draw near.
%</"d0254">
%<*"d0255">
Deacon, in a low voice: Amen.
%</"d0255">
%<*"d0256">
December 25th. The Nativity according to the flesh of our Lord and God and Saviour, Jesus Christ, and until the Close on the 31st.
%</"d0256">
%<*"d0257">
December 25th. The Nativity of our Lord and God and Saviour, Jesus Christ, and until the Leave-taking on December 31st.
%</"d0257">
%<*"d0258">
December 9th. The Conception of the Mother of God by St Anne.
%</"d0258">
%<*"d0259">
Dejection took hold of me because of sinners who abandon your law. Alleluia.
%</"d0259">
%<*"d0260">
Do not forsake me, Lord; my God, do not go far from me.
%</"d0260">
%<*"d0261">
Dread Champion who cannot be put to shame, do not despise our petitions, O Good One. All-praised Mother of God establish the commonwealth of the Orthodox, save your people and give them victory from heaven, for you gave birth to God, O only blessed one.
%</"d0261">
%<*"d0262">
During Eastertide (in some places, always):
%</"d0262">
%<*"d0263">
During Eastertide:
%</"d0263">
%<*"d0264">
During the rest of Thomas Week.
%</"d0264">
%<*"d0265">
Dynamis.
%</"d0265">
%<*"d0266">
ENTRANCE WITH THE HOLY GOSPEL
%</"d0266">
%<*"d0267">
Each time the Deacon concludes:
%</"d0267">
%<*"d0268">
Eis Polla Eti (Many Years)
%</"d0268">
%<*"d0269">
Ekphonesis.
%</"d0269">
%<*"d0270">
Eternal your memory, our brother/sister, worthy of blessedness and ever-remembered.
%</"d0270">
%<*"d0271">
Everlasting Memory. (three times).
%</"d0271">
%<*"d0272">
Everlasting be the memory, everlasting be the memory, everlasting be his (her) memory.
%</"d0272">
%<*"d0273">
Everlasting be the memory, everlasting be the memory, everlasting be his/her memory.
%</"d0273">
%<*"d0274">
Everlasting your memory, our brother (brothers), worthy of blessedness and ever-remembered.
%</"d0274">
%<*"d0275">
Everlasting your memory, our sister (brothers and sisters), worthy of blessedness and ever-remembered.
%</"d0275">
%<*"d0276">
Every dawn I meditated upon you, for you became my helper, and in the shelter of your wings I shall rejoice.
%</"d0276">
%<*"d0277">
Every day I will bless you, * and praise your name for ever * and to the ages of ages.
%</"d0277">
%<*"d0278">
FIRST ANTIPHON
%</"d0278">
%<*"d0279">
FIRST PRAYER OF THE FAITHFUL
%</"d0279">
%<*"d0280">
FOR KOLLYVA
%</"d0280">
%<*"d0281">
Family: Amen.
%</"d0281">
%<*"d0282">
February 2nd. The Meeting of the Lord, and until the Close.
%</"d0282">
%<*"d0283">
Fervour of the Holy Spirit. Amen.
%</"d0283">
%<*"d0284">
Finally the Priest cuts a particle for himself, saying:
%</"d0284">
%<*"d0285">
First Prayer.
%</"d0285">
%<*"d0286">
For 4 Verses.
%</"d0286">
%<*"d0287">
For 4 verses.
%</"d0287">
%<*"d0288">
For 6 verses.
%</"d0288">
%<*"d0289">
For 8 verses.
%</"d0289">
%<*"d0290">
For I have become like a wineskin in the frost; I have not forgotten your statutes. Have mercy on your servant.
%</"d0290">
%<*"d0291">
For all devout and Orthodox Christians, let us pray to the Lord.
%</"d0291">
%<*"d0292">
For all the Powers of heaven praise you, and to you we give glory, to the Father, the Son and the Holy Spirit, now and for ever, and to the ages of ages.
%</"d0292">
%<*"d0293">
For favourable weather, an abundance of the fruits of the earth, and temperate seasons, let us pray to the Lord.
%</"d0293">
%<*"d0294">
For his life is taken away from the earth.
%</"d0294">
%<*"d0295">
For infants (up to the age of 2) we do not say the petitions, but only
%</"d0295">
%<*"d0296">
For it is you who bless and sanctify all things, Christ our God, and to you we give glory, together with your Father who is without beginning, and your all-holy, good and life-giving Spirit, now and for ever, and to the ages of ages. Amen.
%</"d0296">
%<*"d0297">
For one person
%</"d0297">
%<*"d0298">
For only Yours is the kingdom of heaven and to you we ascribe glory, together with Your Eternal Father, and your most holy, good and life-giving Spirit, now and for ever, and unto the ages of ages.
%</"d0298">
%<*"d0299">
For our Archbishop (N) and our Bishop (N), the honorable Presbytery, the Deaconate in Christ, and all the clergy and the laity, let us pray to the Lord.
%</"d0299">
%<*"d0300">
For our Archbishop N, for the honoured order of presbyters, for the diaconate in Christ, for all the clergy and the people, let us pray to the Lord.
%</"d0300">
%<*"d0301">
For our Archbishop N., for his Priesthood, for help, continuance, peace, health, salvation and for the work of his hands, let us pray to the Lord.
%</"d0301">
%<*"d0302">
For our Archbishop N., for the honoured order of presbyters, for the diaconate in Christ, for all the clergy and the people, let us pray to the Lord.
%</"d0302">
%<*"d0303">
For our country, the president, and all those in public service, let us pray to the Lord.
%</"d0303">
%<*"d0304">
For our deliverance from all affliction, wrath, danger and constraint, let us pray to the Lord.
%</"d0304">
%<*"d0305">
For our deliverance from all distress, anger, danger and want, let us pray to the Lord.
%</"d0305">
%<*"d0306">
For remembrance and forgiveness of sins of the blessed founders of this holy house.
%</"d0306">
%<*"d0307">
For the 1st, 2nd Plagal, Grave and 4th Plagal Modes (Modes 4, 5, 6, 7) the following
%</"d0307">
%<*"d0308">
For the 1st, 2nd, 3rd and 4th Modes the following
%</"d0308">
%<*"d0309">
For the Prayers before Communion, see page
%</"d0309">
%<*"d0310">
For the epigonation, if the Priest has the right to wear it:
%</"d0310">
%<*"d0311">
For the epitrachelion:
%</"d0311">
%<*"d0312">
For the girdle:
%</"d0312">
%<*"d0313">
For the holy and sacred Offering of the precious gifts, let us pray to the Lord. Lord, have mercy.
%</"d0313">
%<*"d0314">
For the left cuff:
%</"d0314">
%<*"d0315">
For the orarion:
%</"d0315">
%<*"d0316">
For the peace from above and for the salvation of our souls, let us pray to the Lord.
%</"d0316">
%<*"d0317">
For the peace from above and the salvation of our souls, let us pray to the Lord.
%</"d0317">
%<*"d0318">
For the peace from on high and for the salvation of our souls, let us pray to the Lord.
%</"d0318">
%<*"d0319">
For the peace of the whole world, for the welfare of the holy Churches of God, and for the union of all, let us pray to the Lord.
%</"d0319">
%<*"d0320">
For the pectoral Cross, if the Priest has the right to wear it:
%</"d0320">
%<*"d0321">
For the phelonion:
%</"d0321">
%<*"d0322">
For the right cuff:
%</"d0322">
%<*"d0323">
For their and our deliverance from all affliction, wrath, danger and constraint, let us pray to the Lord.
%</"d0323">
%<*"d0324">
For thine is the kingdom and the power and the glory of the Father and of the Son and of the Holy Spirit, now and ever, and to the ages of ages.
%</"d0324">
%<*"d0325">
For this city, for every city, town and village, and for the faithful who dwell in them, let us pray to the Lord.
%</"d0325">
%<*"d0326">
For this holy house, and for those who enter it with faith, reverence and the fear of God, let us pray to the Lord.
%</"d0326">
%<*"d0327">
For those who travel by land, air or water, for the sick, the suffering, for those in captivity, and for their safety and salvation, let us pray to the Lord.
%</"d0327">
%<*"d0328">
For to You belong all glory, honor, and worship, to the Father and the Son and the Holy Spirit, now and forever and to the ages of ages.
%</"d0328">
%<*"d0329">
For to you belong all glory, honor and worship, to the Father and to the Son and to the Holy Spirit, now and ever, and to the ages of ages.
%</"d0329">
%<*"d0330">
For to you belong all glory, honour and worship, Father, Son and Holy Spirit, now and for ever, and to the ages of ages.
%</"d0330">
%<*"d0331">
For to you belong all glory, honour and worship, to the Father, the Son and the Holy Spirit, now and for ever, and to the ages of ages.
%</"d0331">
%<*"d0332">
For to you belong all glory, honour and worship, to the Father, the Son and the Holy Spirit, now and for ever, and to the ages of ages. Amen.
%</"d0332">
%<*"d0333">
For two or more people:
%</"d0333">
%<*"d0334">
For we are about to receive the King of all, invisibly escorted by the angelic hosts. Alleluia, alleluia, alleluia.
%</"d0334">
%<*"d0335">
For you alone are holy, * you alone are Lord, * Jesus Christ, * to the glory of God the Father. Amen.
%</"d0335">
%<*"d0336">
For you are a merciful God who love mankind, and to you we ascribe glory, to the Father and to the Son and to the Holy Spirit, now and ever, and to the ages of ages.
%</"d0336">
%<*"d0337">
For you are blessed and glorified to the ages. Amen. (three times).
%</"d0337">
%<*"d0338">
For you are our God and to you we give glory, to the Father, the Son and the Holy Spirit, now and for ever, and to the ages of ages. Amen.
%</"d0338">
%<*"d0339">
For you are our God, and to you we give glory, to the Father, the Son and the Holy Spirit, now and for ever, and to the ages of ages.
%</"d0339">
%<*"d0340">
For you are our God, and to you we give glory, together with your Father who has no beginning, and your all-holy, good and life-giving Spirit, now and for ever, and to the ages of ages. Amen.
%</"d0340">
%<*"d0341">
For you are our sanctification, and to you we give glory, Father, Son and Holy Spirit, now and for ever, and to the ages of ages.
%</"d0341">
%<*"d0342">
For you are the King of peace and the Saviour of our souls, and to you we give glory, to the Father, the Son and the Holy Spirit, now and for ever, and to the ages of ages.
%</"d0342">
%<*"d0343">
For you are the resurrection, the life and the repose of your departed servant [name], who has fallen asleep, O Christ our God, and to you we ascribe glory, together with your eternal Father and your all-holy, good and life-giving Spirit, now and for ever, and to the ages of ages.
%</"d0343">
%<*"d0344">
For you are the resurrection, the life and the repose of your servant N., who has fallen asleep, Christ our God, and to you we give glory, together with your Father who is without beginning, and your all-holy, good and life-giving Spirit, now and for ever, and to the ages of ages.
%</"d0344">
%<*"d0345">
For you are the resurrection, the life and the repose of your servant(s) N. (and N.), who has (have) fallen asleep, Christ our God, and to you we give glory, together with your Father who is without beginning, and your all-holy, good and life-giving Spirit, now and for ever, and to the ages of ages.
%</"d0345">
%<*"d0346">
For you are the sanctification of our souls and bodies, and to you we ascribe glory and thanksgiving and worship, together with your beginning-less Father and your all-holy and good and life-giving Spirit, now and ever, and to the ages of ages.
%</"d0346">
%<*"d0347">
For you, O God, are good and love mankind, and to you we give glory, Father, Son and Holy Spirit, now and for ever, and to the ages of ages.
%</"d0347">
%<*"d0348">
For you, O God, are good and love mankind, and to you we give glory, to the Father and to the Son and to the Holy Spirit, now and for ever, and to the ages of ages.
%</"d0348">
%<*"d0349">
For you, O God, are good and love mankind, and to you we give glory, to the Father, the Son and the Holy Spirit, now and for ever, and to the ages of ages. Amen.
%</"d0349">
%<*"d0350">
For you, O God, are merciful and love mankind, and to you we give glory, Father, Son and Holy Spirit, now and for ever, and to the ages of ages.
%</"d0350">
%<*"d0351">
For you, O God, are merciful and love mankind, and to you we give glory, to the Father, the Son and the Holy Spirit, now and for ever, and to the ages of ages.
%</"d0351">
%<*"d0352">
For you, O God, are merciful and you love mankind, and to you we give glory, to the Father and to the Son and to the Holy Spirit, now and for ever, and to the ages of ages.
%</"d0352">
%<*"d0353">
For you, O God, are merciful, and love mankind, and to you we give glory, Father, Son and Holy Spirit, now and for ever, and to the ages of ages.
%</"d0353">
%<*"d0354">
For you, O God, are merciful, and love mankind, and to you we give glory, to the Father, the Son and the Holy Spirit, now and for ever, and to the ages of ages.
%</"d0354">
%<*"d0355">
For you, our God, are holy, and to you we give glory, Father, Son and Holy Spirit, now and for ever,
%</"d0355">
%<*"d0356">
For yours is the Kingdom, the power and the glory, of the Father, the Son and the Holy Spirit, now and for ever, and to the ages of ages.
%</"d0356">
%<*"d0357">
For yours is the kingdom, the power and the glory, Father, Son and Holy Spirit, now and for ever, and to the ages of ages.
%</"d0357">
%<*"d0358">
For yours is the kingdom, the power and the glory, of the Father, the Son and the Holy Spirit, now and for ever, and to the ages of ages.
%</"d0358">
%<*"d0359">
For yours is the might and yours is the kingdom, the power and the glory, of the Father, the Son and the Holy Spirit, now and for ever, and to the ages of ages.
%</"d0359">
%<*"d0360">
For yours is the might, and yours is the kingdom, the power and the glory, Father, Son and Holy Spirit, now and for ever, and to the ages of ages.
%</"d0360">
%<*"d0361">
For yours it is to have mercy and to save us, O God, our God, and to you we give glory, to Father, Son and holy Spirit, now and forever, and to the ages of ages.
%</"d0361">
%<*"d0362">
Forgive me, or similar words.
%</"d0362">
%<*"d0363">
From January 2nd until the Eve of Theophany (January 5th).
%</"d0363">
%<*"d0364">
From July 27th to the Close of the Transfiguration on August 13th.
%</"d0364">
%<*"d0365">
From November 8th until the Leave-taking of the Entry on the 25th.
%</"d0365">
%<*"d0366">
From my youth up many passions make war on me: but you, O Saviour, help me and save me. (x2)
%</"d0366">
%<*"d0367">
From the 1st Sunday of the Fast, until the Saturday of the Akathist.
%</"d0367">
%<*"d0368">
From the Close of Theophany until the Leave-taking of the Meeting.
%</"d0368">
%<*"d0369">
From the Close of the Dormition until September 12th.
%</"d0369">
%<*"d0370">
From the Close of the Meeting to the beginning of the Triodion. (Except in churches dedicated to the Mother of God, which should use the appropriate Kontakion.)
%</"d0370">
%<*"d0371">
From the Leave-taking of the Cross until November 7th. (Except in churches dedicated to the Mother of God, which should use the appropriate Kontakion.)
%</"d0371">
%<*"d0372">
From the Leave-taking of the Entry until December 24th.
%</"d0372">
%<*"d0373">
From the home until reaching the temple the choir sing slowly:
%</"d0373">
%<*"d0374">
From the temple until the grave is sung: "Today is the day of Resurrection..."
%</"d0374">
%<*"d0375">
Fullness of the Holy Spirit.
%</"d0375">
%<*"d0376">
Give rest, O God, to your servant(s), and settle him (her/them) in Paradise, where the choirs of the Saints and all the Just shine out like beacons; give rest to your servant(s) who has (have) fallen asleep, overlooking all his (her/their) offences.
%</"d0376">
%<*"d0377">
Give the order.
%</"d0377">
%<*"d0378">
Glory be to You, O Lord, glory be to You.
%</"d0378">
%<*"d0379">
Glory be to You, O our God, glory be to You.
%</"d0379">
%<*"d0380">
Glory be to the Father, and to the Son, and to the Holy Spirit, now and for ever, and unto the ages of ages. Amen.
%</"d0380">
%<*"d0381">
Glory to God in the highest, and on earth peace; good will among men.
%</"d0381">
%<*"d0382">
Glory to God in the highest, and on earth peace; good will among men. (three times).
%</"d0382">
%<*"d0383">
Glory to the Father and to the Son and to the Holy Spirit, now and ever and to the ages of ages. Amen.
%</"d0383">
%<*"d0384">
Glory to the Father and to the Son and to the Holy Spirit, now and ever, and to the ages of ages. Amen.
%</"d0384">
%<*"d0385">
Glory to the Father and to the Son and to the Holy Spirit.
%</"d0385">
%<*"d0386">
Glory to the Father and to the Son and to the holy Spirit.
%</"d0386">
%<*"d0387">
Glory to the Father and to the Son and to the holy Spirit. Both now and for ever, and to the ages of ages. Amen.
%</"d0387">
%<*"d0388">
Glory to the Father, and to the Son, and to the Holy Spirit, both now and for ever, and to the ages of ages. Amen
%</"d0388">
%<*"d0389">
Glory to the Father, and to the Son, and to the Holy Spirit, both now and for ever, and to the ages of ages. Amen.
%</"d0389">
%<*"d0390">
Glory to the Father, and to the Son, and to the Holy Spirit.
%</"d0390">
%<*"d0391">
Glory to the Father, and to the Son, and to the Holy Spirit. Alleluia.
%</"d0391">
%<*"d0392">
Glory to the Father, and to the Son, and to the Holy Spirit. Both now and for ever, and to the ages of ages. Amen.
%</"d0392">
%<*"d0393">
Glory to the Father, and to the Son, and to the Holy Spirit. Both now and for ever, and to the ages of ages. Amen. the Kontakion of the Feast or Season.
%</"d0393">
%<*"d0394">
Glory to the Father, and to the Son, and to the Holy Spirit. Have mercy on your servant.
%</"d0394">
%<*"d0395">
Glory to the Father, and to the Son, and to the Holy Spirit. Of the Patron Saint.
%</"d0395">
%<*"d0396">
Glory to the Father, and to the Son, and to the Holy Spirit. Of the Saint of the Church.
%</"d0396">
%<*"d0397">
Glory to the Father, and to the Son, and to the Holy Spirit; both now and for ever, and to the ages of ages. Amen
%</"d0397">
%<*"d0398">
Glory to the Father, and to the Son, and to the Holy Spirit; both now and for ever, and to the ages of ages. Amen.
%</"d0398">
%<*"d0399">
Glory to the Father, and to the Son, and to the Holy Spirit; both now and for ever, and to the ages of ages. Amen. Lord, have mercy. (three times). Holy Master, bless.
%</"d0399">
%<*"d0400">
Glory to the Father, and to the Son, and to the Holy Spirit; both now and for ever, and to the ages of ages. Amen. Lord, have mercy. (three times). Holy father, give the blessing.
%</"d0400">
%<*"d0401">
Glory to the Father, and to the Son, and to the Holy Spirit; both now and for ever, and to the ages of ages. Amen. Lord, have mercy. Lord, have mercy. Lord, have mercy. Holy Master, bless.
%</"d0401">
%<*"d0402">
Glory to the holy, consubstantial, life-giving and undivided Trinity, always, now and for ever, and to the ages of ages.
%</"d0402">
%<*"d0403">
Glory to you who have shown us the light.
%</"d0403">
%<*"d0404">
Glory to you, Christ God, boast of Apostles, joy of Martyrs whose preaching was the consubstantial Trinity.
%</"d0404">
%<*"d0405">
Glory to you, Christ God, our hope, glory to you.
%</"d0405">
%<*"d0406">
Glory to you, Lord! Glory to you!
%</"d0406">
%<*"d0407">
Glory to you, Lord, glory to you!
%</"d0407">
%<*"d0408">
Glory to you, Lord, glory to you.
%</"d0408">
%<*"d0409">
Glory to you, O Christ our God, crown of all Apostles. The martyrs did with joy proclaim, that the Trinity was One in essence evermore.
%</"d0409">
%<*"d0410">
Glory to you, O Lord, glory to you!
%</"d0410">
%<*"d0411">
Glory to you, O Lord, glory to you.
%</"d0411">
%<*"d0412">
Glory to you, our God, glory to you.
%</"d0412">
%<*"d0413">
Glory to you. O Lord, Glory to you.
%</"d0413">
%<*"d0414">
Glory to you. O Lord, glory to you.
%</"d0414">
%<*"d0415">
Glory.
%</"d0415">
%<*"d0416">
Glory. Both now.
%</"d0416">
%<*"d0417">
Glory. Triadikon.
%</"d0417">
%<*"d0418">
Glory\u2026 Now and for ever\u2026
%</"d0418">
%<*"d0419">
God, cleanse me a sinner.
%</"d0419">
%<*"d0420">
God, cleanse me a sinner. (three times).
%</"d0420">
%<*"d0421">
God, cleanse me, a sinner and have mercy on me.
%</"d0421">
%<*"d0422">
God, our God, who sent forth the heavenly Bread, the food of the whole world, our Lord and God Jesus Christ, as our Saviour, and Redeemer and Benefactor, to bless and sanctify us; bless this Offering, and receive it on your altar above the heavens. In your goodness and love for mankind be mindful of those who have offered it, and those for whom they have offered it; and as we celebrate your divine mysteries keep us without condemnation. For sanctified and glorified is your all-honoured and majestic name, of the Father, the Son and the Holy Spirit, now and for ever, and to the ages of ages. Amen.
%</"d0422">
%<*"d0423">
Grant this, O Lord.
%</"d0423">
%<*"d0424">
Grant this, O Lord. And so after each of the following petitions.
%</"d0424">
%<*"d0425">
Grant this, O Lord. And so after each petition.
%</"d0425">
%<*"d0426">
Grant us, O Lord.
%</"d0426">
%<*"d0427">
Grant, Lord, * this day * to keep us without sin.
%</"d0427">
%<*"d0428">
Grave Tone
%</"d0428">
%<*"d0429">
Great are the achievements of faith! In the fountain of flame, as by water of rest, the three holy Youths rejoiced; and the Prophet Daniel was revealed shepherding lions like sheep. At their intercessions, Christ God, save our souls.
%</"d0429">
%<*"d0430">
Greater in honour than the Cherubim * and beyond compare more glorious than the Seraphim, * without corruption * you gave birth to God the Word; * truly the Mother of God, * we magnify you.
%</"d0430">
%<*"d0431">
Greater in honour than the Cherubim, and beyond compare more glorious than the Seraphim, without corruption you gave birth to God the Word; truly the Mother of God, we magnify you.
%</"d0431">
%<*"d0432">
HOLY COMMUNION OF THE CLERGY AND PEOPLE
%</"d0432">
%<*"d0433">
Hail, full of grace, Virgin Mother of God, for from you there dawned the Sun of righteousness, Christ our God, who enlightens those in darkness. Be glad too, righteous Elder, for you received in your embrace the Liberator of our souls, who grants us also resurrection.
%</"d0433">
%<*"d0434">
Hail, honoured one, who bore God in the flesh for the salvation of all; through you the human race has found salvation; through you may we find Paradise, O pure and blessed Mother of God.
%</"d0434">
%<*"d0435">
Hasten to help me, Lord of my salvation.
%</"d0435">
%<*"d0436">
Have mercy on me O God, in your great mercy; according to the fullness of your compassion blot out my offence. Wash me thoroughly from my wickedness, and cleanse me from my sin. For I acknowledge my wickedness, and my sin is ever before me. Against you only I have sinned and done what is evil in your sight, that you may be justified in your words, and win when you are judged. For see, in wickedness I was conceived, and in sin my mother bore me. For see, you have loved truth: you have shown me the hidden and secret things of your wisdom. You will sprinkle me with hyssop and I shall be cleansed; you will wash me, and I shall be made whiter than snow. You will make me hear of joy and gladness; the bones which have been humbled will rejoice. Turn away your face from my sins, and blot out all my iniquities. Create a clean heart in me, O God, and renew a right Spirit within me. Do not cast me out from your presence, and do not take your Holy Spirit from me. Give me back the joy of your salvation, and establish me with a sovereign Spirit. I will teach transgressors your ways, and sinners will turn to you again. O God, the God of my salvation, deliver me from bloodshed, and my tongue will rejoice at your justice. Lord, you will open my lips: and my mouth will declare your praise. For if you had wanted a sacrifice, I would have given it; you will not take pleasure in burnt offerings. A sacrifice to God is a broken spirit; a broken and a humbled heart God will not despise. Do good to Sion, Lord, in your good pleasure; and let the walls of Jerusalem be rebuilt. Then you will be well pleased with a sacrifice of justice, oblation and whole burnt offerings. Then they will offer calves upon your altar.
%</"d0436">
%<*"d0437">
Have mercy on us, Lord, have mercy on us; for we sinners, lacking all defence, offer you, as our Master, this supplication: have mercy on us.
%</"d0437">
%<*"d0438">
Have mercy on us, O God, according to your great mercy, we pray to you, hear us and have mercy.
%</"d0438">
%<*"d0439">
Have mercy on us, O God, according to your great mercy, we pray you, hear and have mercy.
%</"d0439">
%<*"d0440">
Have mercy on us, O God, according to your great mercy. We pray you, hear and have mercy.
%</"d0440">
%<*"d0441">
Have mercy on us, O God, in accordance with your great mercy, we pray you, hear and have mercy.
%</"d0441">
%<*"d0442">
Have mercy on us, O God, in accordance with your great mercy; we pray you hear and have mercy.
%</"d0442">
%<*"d0443">
Have mercy upon us, O God, according to Your great mercy; we pray You, hear us and have mercy.
%</"d0443">
%<*"d0444">
Have mercy upon us, O God, according to your great mercy; we pray you, hear us and have mercy.
%</"d0444">
%<*"d0445">
Have you united yourself to Christ?
%</"d0445">
%<*"d0446">
Having seen the Resurrection of Christ..., and Psalm 50, excluding the last two verses which begin, Do good, Lord, to Sion. If it is not a Sunday he says,
%</"d0446">
%<*"d0447">
Having venerated the icons, standing again in front of the holy doors, they bow their heads, uncovered, and the Deacon says:
%</"d0447">
%<*"d0448">
He arranges them on the Paten in the form of a Cross, thus:
%</"d0448">
%<*"d0449">
He blesses each vestment in turn and puts it on, saying,
%</"d0449">
%<*"d0450">
He drinks three times from the Chalice, wipes his lips and the Chalice with the cloth, and kisses the Chalice, saying:
%</"d0450">
%<*"d0451">
He is and will be.
%</"d0451">
%<*"d0452">
He lays the Lamb inverted (with the seal downward) on the paten.
%</"d0452">
%<*"d0453">
He puts away the censer and says to the Deacon:
%</"d0453">
%<*"d0454">
He says this three times as he makes the sign of the Cross on each of them.
%</"d0454">
%<*"d0455">
He takes a portion of the precious Body of Christ, from the part stamped with the letters XC, and says:
%</"d0455">
%<*"d0456">
He takes the hot water and says to the Priest:
%</"d0456">
%<*"d0457">
He then censes the whole sanctuary and church as usual.
%</"d0457">
%<*"d0458">
He turns the Lamb over so that the seal is uppermost.
%</"d0458">
%<*"d0459">
Hear my voice, O Lord, in accordance with your mercy; in accordance with your judgement give me life. Alleluia.
%</"d0459">
%<*"d0460">
Hear us, O God, our Saviour, the hope of all the ends of the earth and of those far off at sea; and have pity, have pity, Master, on our sins, and have mercy on us. For you, O God, are merciful, and love mankind, and to you we give glory, Father, Son and Holy Spirit, now and for ever, and to the ages of ages.
%</"d0460">
%<*"d0461">
Heavenly King, Advocate, Spirit of truth, present everywhere, filling all things, Treasury of blessings and Giver of life, come and dwell in us, cleanse us from every stain, and, O Good One, save our souls.
%</"d0461">
%<*"d0462">
Help us, save us, have mercy on us and keep us, O God, by your grace.
%</"d0462">
%<*"d0463">
Help us, save us, have mercy on us, and keep us, O God, by your grace.
%</"d0463">
%<*"d0464">
Holy God, Holy Mighty, Holy Immortal, have mercy on us. (3)
%</"d0464">
%<*"d0465">
Holy God, Holy Strong, Holy Immortal, have mercy on us (three times).
%</"d0465">
%<*"d0466">
Holy God, Holy Strong, Holy Immortal, have mercy on us (x3).
%</"d0466">
%<*"d0467">
Holy God, Holy Strong, Holy Immortal, have mercy on us.
%</"d0467">
%<*"d0467a">
Holy God, Holy Strong, Holy Immortal, have mercy on us.(
%</"d0467a">
%<*"d0468">
Holy God, Holy Strong, Holy Immortal, have mercy on us. (three times)
%</"d0468">
%<*"d0469">
Holy God, Holy Strong, Holy Immortal, have mercy on us. (three times).
%</"d0469">
%<*"d0470">
Holy God, Holy Strong, Holy Immortal, have mercy on us. (x3)
%</"d0470">
%<*"d0471">
Holy Immortal, have mercy on us.
%</"d0471">
%<*"d0472">
Holy Martyrs, who fought the good fight and were crowned, intercede with the Lord to have mercy on our souls.
%</"d0472">
%<*"d0473">
Holy is the Lord our God.
%</"d0473">
%<*"d0474">
How shall I, the unworthy, enter among the splendours of your Saints? For if I dare to enter with them into the bridal chamber, my dress convicts me, for it is not a wedding garment, and I shall be bound and cast out by the Angels. Cleanse the stain of my soul, Lord, and save me, as you love humankind.
%</"d0474">
%<*"d0475">
I am a companion of all who fear you, and who keep your commandments. Alleluia.
%</"d0475">
%<*"d0476">
I am an image of your ineffable glory, though I bear the marks of offences; take pity on your creature, Master, and with compassion cleanse me; and give me the longed-for fatherland, making me once again a citizen of Paradise.
%</"d0476">
%<*"d0477">
I am young and despised; I have not forgotten your statutes. Alleluia.
%</"d0477">
%<*"d0478">
I am yours; save me, for I have sought your statutes. Have mercy on your servant.
%</"d0478">
%<*"d0479">
I believe in one God, * Father, Almighty,* Maker of heaven and earth,* and of all things visible and invisible. And in one Lord, Jesus Christ,* the only-begotten Son of God, * begotten from the Father * before all ages, * Light from Light, * true God from true God, * begotten * not made, * consubstantial with the Father,* through him all things were made. * For our sake and for our salvation * he came down from heaven, * and was incarnate from the Holy Spirit * and the Virgin Mary * and became man. * He was crucified also for us under Pontius Pilate,* and suffered and was buried; * he rose again on the third day, * in accordance with the Scriptures, * and ascended into heaven * and is seated at the right hand of the Father. * He is coming again in glory * to judge the living and the dead, * and his kingdom will have no end. And in the Holy Spirit, * the Lord, * the Giver of life, * who proceeds from the Father, * who together with Father and Son is worshipped * and together glorified; * who spoke through the Prophets. * In one, Holy, * Catholic and Apostolic Church. * I confess one Baptism * for the forgiveness of sins. * I await the resurrection of the dead * and the life of the age to come. Amen.
%</"d0479">
%<*"d0480">
I did not turn aside from your judgements, for you gave me your law. Have mercy on your servant.
%</"d0480">
%<*"d0481">
I have foolishly run away, O Father, from your glory; I have squandered in evil deeds the riches you entrusted to me; therefore I cry to you in the words of the Prodigal: I have sinned before you, compassionate Father: take me now repentant and make me as one of your hired servants.
%</"d0481">
%<*"d0482">
I have gone astray like a lost sheep; seek out your servant, because I have not forgotten your commandments.
%</"d0482">
%<*"d0483">
I inclined my heart to do your statutes for ever, because of the recompense. Have mercy on your servant.
%</"d0483">
%<*"d0484">
I lay down and slept; I awoke because the Lord will assist me.
%</"d0484">
%<*"d0485">
I thank you, Lord, my God, because you have not rejected me a sinner, but have counted me worthy to be a communicant of your Holy Things. I thank you, because you have counted me, the unworthy, worthy to share in your most pure and heavenly gifts. But, Master, Lover of mankind, who died for our sake and rose again, and gave us these your awe-inspiring and life-giving Mysteries, for the well-being and hallowing of our souls and bodies, grant that these gifts may bring me also healing of soul and body, the repelling of every adversary, the enlightenment of the eyes of my heart, peace of my spiritual powers, faith unashamed, love without pretence, fullness of wisdom, the keeping of your commandments, increase of your divine grace and the gaining of your Kingdom; that preserved through them by your sanctification, I may always remember your grace, and no longer live for myself but for you, our Master and Benefactor. And so, when I leave this present life in the hope of life eternal, I shall find everlasting repose where the sound of those who feast is unceasing, and the delight of those who see the ineffable beauty of your face is unbounded. For you are the true desire and the inexpressible joy of those who love you, Christ our God, and all creation hymns you to the ages. Amen.
%</"d0485">
%<*"d0486">
I will remember your name throughout all generations.
%</"d0486">
%<*"d0487">
IN THE CHURCH
%</"d0487">
%<*"d0488">
IN THE EIGHT MODES
%</"d0488">
%<*"d0489">
If it is Sunday the second choir chants the Resurrection Troparion.
%</"d0489">
%<*"d0490">
If it is a feast of the Lord or of the Mother of God, or its leave-taking, the Irmos of the 9th Ode of the Canon is sung instead.
%</"d0490">
%<*"d0491">
If it is a feast of the Lord, or of the Mother of God, or its afterfeast or leavetaking, the Antiphons of the feast are sung; otherwise the Typika are used.
%</"d0491">
%<*"d0492">
If someone wishes to come after me let him deny himself, take up his cross, and follow me.
%</"d0492">
%<*"d0493">
If the child is older then we say:
%</"d0493">
%<*"d0494">
If there is more than one Deacon they also exchange the Kiss with one another in the same way.
%</"d0494">
%<*"d0495">
If there is more than one Priest, at the end of each stasis the Deacon will say "Let us pray unto the Lord"  and every priest according to rank will say "For only Yours is the kingdom of heaven..."
%</"d0495">
%<*"d0496">
Immediately after the Prayer behind the Ambo the Singers begin the Troparion of the Feast or Saint and the Priest and Deacon come from the Sanctuary as above. At the end, after "The rich have become poor ...," the Singers begin "Blessed be the name of the Lord..." as usual; and the Liturgy ends in the normal way. If there is a Memorial Service for the Departed it takes place immediately after "The rich have become poor...," and is followed by "Blessed be the name of the Lord."
%</"d0496">
%<*"d0497">
In Commemoration of a Saint
%</"d0497">
%<*"d0498">
In a low voice he says, if it is Sunday,
%</"d0498">
%<*"d0499">
In every place of his dominion, bless the Lord, O my soul!
%</"d0499">
%<*"d0500">
In giving birth you retained your virginity; in falling asleep you did not abandon the world, O Mother of God; you passed over into life, for you are the Mother of Life; and by your prayers you deliver our souls from death.
%</"d0500">
%<*"d0501">
In his humiliation judgement was denied him.
%</"d0501">
%<*"d0502">
In honour and memory of our most blessed and glorious Lady, Mother of God and Ever-Virgin Mary; at whose prayers, Lord, accept this sacrifice at your altar above the heavens.
%</"d0502">
%<*"d0503">
In peace let us pray to the Lord.
%</"d0503">
%<*"d0504">
In peace, let us pray to the Lord.
%</"d0504">
%<*"d0505">
In remembrance of our Lord and God and Saviour, Jesus Christ (three times).
%</"d0505">
%<*"d0506">
In the divine Transfiguration all mortal nature divinely shines forth today, as with gladness it cries: \u2018Christ is transfigured, saving all'.
%</"d0506">
%<*"d0507">
In the name of the Lord.
%</"d0507">
%<*"d0508">
In you, Mother, was preserved unimpaired that which is according to God's image; for you took up the Cross and followed Christ, and by your deeds you taught us to despise the flesh, for it passes away, but to care for the soul, which is a thing immortal. And therefore your spirit, holy Mary, rejoices with the Angels.
%</"d0508">
%<*"d0509">
In your justice hear me, O Lord; and do not enter into judgement with your servant. (x2)
%</"d0509">
%<*"d0510">
In your repose where all your saints find rest, give rest, O Lord, to the soul of your servant, for you alone are immortal.
%</"d0510">
%<*"d0511">
In your repose where all your saints find rest, give rest, O Lord, to the soul(s) of your servant(s), for you alone are immortal.
%</"d0511">
%<*"d0512">
Incline my heart to your testimonies, and not to covetousness. Alleluia.
%</"d0512">
%<*"d0513">
Incline your ear, O Lord, and hear us, you who accepted to be baptized in the Jordan and sanctified the waters; and bless all of us who, through the bowing of our heads, express our obedience to you; grant that we may be filled with your sanctification through the drinking and sprinkling of this water, and may it bring us, O Lord, health of soul and body.
%</"d0513">
%<*"d0514">
Instead of "Let us give a parting farewell..." the choir sing:
%</"d0514">
%<*"d0515">
Is polla eti despota!
%</"d0515">
%<*"d0516">
Isaiah rejoice, for the Virgin did conceive. And bore the son whom thou called Emmanuel, who is both God and Man, and "Day-at-the-Dawn"(Oriental) is His name. Which is magnified with the Virgin hymned as blessed.
%</"d0516">
%<*"d0517">
Isaias dance: the Virgin has conceived and given birth to a Son, Emmanuel, who is both God and man. Orient is his name, whom we magnify as we call the Virgin blessed.
%</"d0517">
%<*"d0518">
It is impossible for humans to see God, on whom the ranks of Angels dare not gaze. But through you, All-pure one, the incarnate Word has been seen by mortals. As we magnify him, with the heavenly armies we call you blessed.
%</"d0518">
%<*"d0519">
It is time for the Lord to act; they have cast your law to the winds. Have mercy on your servant.
%</"d0519">
%<*"d0520">
It is truly right to call you blessed, who gave birth to God, ever-blessed and most pure, and Mother of our God. Greater in honour than the Cherubim and beyond compare more glorious than the Seraphim, without corruption you gave birth to God the Word; truly the Mother of God, we magnify you.
%</"d0520">
%<*"d0521">
January 1st. The Circumcision of Our Lord, Jesus Christ and Commemoration of St Basil the Great.
%</"d0521">
%<*"d0522">
January 6th. The holy Theophany of our Lord, Jesus Christ, and until the Close on the 14th.
%</"d0522">
%<*"d0523">
January 6th. The holy Theophany of our Lord, and until the Leave-taking on the 14th.
%</"d0523">
%<*"d0524">
Jesus having risen from the tomb as He foretold, has granted unto us, life eternal and great mercy.
%</"d0524">
%<*"d0525">
Joachim and Anna were set free from the reproach of childlessness, and Adam and Eve from the corruption of death, by your holy nativity, O Most Pure. Delivered from the guilt of offences, your people also celebrate the feast as they cry to you: The barren woman bears the Mother of God, the sustainer of our life.
%</"d0525">
%<*"d0526">
KONTAKIA FOR THE SEASONS OF THE YEAR
%</"d0526">
%<*"d0527">
Kontakion of Saint John Chrysostom.
%</"d0527">
%<*"d0528">
Kontakion. Tone 8.
%</"d0528">
%<*"d0529">
LITANY OF FERVENT SUPPLICATION
%</"d0529">
%<*"d0530">
LITANY OF PEACE
%</"d0530">
%<*"d0531">
LITANY OF THE LORD'S PRAYER
%</"d0531">
%<*"d0532">
LITANY OF THE PRECIOUS GIFTS
%</"d0532">
%<*"d0533">
Let everything in heaven rejoice, let everything on earth be glad, for the Lord has shown strength with his arm; by death he has trampled on death; he has become the first-born from the dead; from the bowels of Hades he has delivered us, and granted to the world his great mercy.
%</"d0533">
%<*"d0534">
Let the Lord God establish his/her soul where the just repose; the mercies of God, the kingdom of the heavens, and the remission of his/her sins, let us beseech of Christ, our immortal King and our God.
%</"d0534">
%<*"d0535">
Let us all say, with all our soul and with all our mind, let us say.
%</"d0535">
%<*"d0536">
Let us arise at the break of dawn, and instead of myrrh, offer a hymn to the Master; and we shall see Christ, the Sun of Righteousness, causing life to dawn on all life.
%</"d0536">
%<*"d0537">
Let us attend
%</"d0537">
%<*"d0538">
Let us attend.
%</"d0538">
%<*"d0539">
Let us be attentive
%</"d0539">
%<*"d0540">
Let us be attentive.
%</"d0540">
%<*"d0541">
Let us bow our heads to the Lord.
%</"d0541">
%<*"d0542">
Let us celebrate the memory of the grandparents of Christ as with faith we ask their help that all those who cry, 'O God be with us, who have glorified them as you were well-pleased' may be delivered from every trouble.
%</"d0542">
%<*"d0543">
Let us complete our morning prayer to the Lord.
%</"d0543">
%<*"d0544">
Let us complete our prayer to the Lord.
%</"d0544">
%<*"d0545">
Let us devoutly hymn the threefold light of the one Godhead as we cry: Holy are you, the Father without beginning, the Son likewise without beginning and the divine Spirit; enlighten us who worship you in faith, and snatch us from the everlasting fire.
%</"d0545">
%<*"d0546">
Let us flee the proud speech of the Pharisee; and let us learn the humility of the Tax-collector, as with groans we cry to the Saviour: Be merciful to us, for you alone are ready to forgive.
%</"d0546">
%<*"d0547">
Let us give heed
%</"d0547">
%<*"d0548">
Let us give heed.
%</"d0548">
%<*"d0549">
Let us go forth in peace.
%</"d0549">
%<*"d0550">
Let us pray to the Lord
%</"d0550">
%<*"d0551">
Let us pray to the Lord.
%</"d0551">
%<*"d0552">
Let us pray unto the Lord.
%</"d0552">
%<*"d0553">
Lifted up on the Cross of your own will, to the new commonwealth that bears your name grant your mercies, Christ God; make your faithful people glad by your power, granting them victories over their enemies; may they have your help in battle: a weapon of peace, an invincible trophy.
%</"d0553">
%<*"d0554">
Lifted up on the Cross of your own will, to the new commonwealth that bears your name grant your mercies, O Christ God; make your faithful people glad by your power, granting them victories over their enemies; may they have your help in battle: a weapon of peace, an invincible trophy.
%</"d0554">
%<*"d0555">
Like a sheep he was led to the slaughter.
%</"d0555">
%<*"d0556">
Likewise the Deacon also remembers those whom he wishes of the living and the dead.
%</"d0556">
%<*"d0557">
Likewise the Deacon, standing in his place, bows and kisses the Cross on his orarion. At a concelebrated Liturgy the Priests here exchange the Kiss of Peace, the senior saying:
%</"d0557">
%<*"d0558">
Litany of Fervent Supplication
%</"d0558">
%<*"d0559">
Litany of Peace
%</"d0559">
%<*"d0560">
Look upon me and have mercy on me, in accordance with the judgement of those who love your name. Alleluia.
%</"d0560">
%<*"d0561">
Lord Almighty, the God of our fathers, we pray you, hear and have mercy.
%</"d0561">
%<*"d0562">
Lord God, * Lamb of God, * Son of the Father, * who takes away the sin of the world, * have mercy on us; * you take away the sins of the world.
%</"d0562">
%<*"d0563">
Lord Have mercy.
%</"d0563">
%<*"d0564">
Lord almighty, the God of our fathers, we pray you, hear and have mercy.
%</"d0564">
%<*"d0565">
Lord have mercy, Lord have mercy, Lord have mercy.
%</"d0565">
%<*"d0566">
Lord have mercy.
%</"d0566">
%<*"d0567">
Lord, Jesus Christ, our God, who blessed the five loaves in the desert, and from them fed five thousand, bless these loaves also, this wheat, wine and oil, and multiply them in this city and in all your world, and sanctify your faithful servants who partake of them. For it is you who bless and sanctify all things, Christ, our God, and to you we give glory, with your Father who is without beginning, and your all-holy, good and life-giving Spirit, now and for ever, and to the ages of ages.
%</"d0567">
%<*"d0568">
Lord, King, * God of heaven, * Father, Almighty, * Lord, only-begotten Son, * Jesus Christ * and holy Spirit.
%</"d0568">
%<*"d0569">
Lord, have mercy (three times after each petition).
%</"d0569">
%<*"d0570">
Lord, have mercy (x3 or x12).
%</"d0570">
%<*"d0571">
Lord, have mercy (x3)
%</"d0571">
%<*"d0572">
Lord, have mercy (x3).
%</"d0572">
%<*"d0573">
Lord, have mercy on us, for in you we have put our trust. Do not be very angry with us, nor remember our iniquities. But look on us now, as you are compassionate, and rescue us from our enemies. For you are our God, and we are your people; we are all the work of your hands, and we have called on your name.
%</"d0573">
%<*"d0574">
Lord, have mercy, Lord, have mercy, Lord, have mercy.
%</"d0574">
%<*"d0575">
Lord, have mercy.
%</"d0575">
%<*"d0576">
Lord, have mercy. (Repeated after each petition)
%</"d0576">
%<*"d0577">
Lord, have mercy. (Three times. And so after the remaining petitions.)
%</"d0577">
%<*"d0578">
Lord, have mercy. (Three times.)
%</"d0578">
%<*"d0579">
Lord, have mercy. (three times).
%</"d0579">
%<*"d0580">
Lord, have mercy. (three times). Glory to the Father, and to the Son, and to the Holy Spirit. Both now and for ever, and to the ages of ages. Amen.
%</"d0580">
%<*"d0581">
Lord, have mercy. (twelve times).
%</"d0581">
%<*"d0582">
Lord, have mercy. (x3)
%</"d0582">
%<*"d0583">
Lord, have mercy. (x40)
%</"d0583">
%<*"d0584">
Lord, have mercy. And so after each of the following petitions.
%</"d0584">
%<*"d0585">
Lord, have mercy. And so after each petition.
%</"d0585">
%<*"d0586">
Lord, have mercy. Lord, have mercy. Lord, have mercy.
%</"d0586">
%<*"d0587">
Lord, have mercy. Lord, have mercy. Lord, have mercy. And so after each petition.
%</"d0587">
%<*"d0588">
Lord, have mercy. Three times after each petition.
%</"d0588">
%<*"d0589">
Lord, now lettest thou thy servant depart in peace, according to thy word; for mine eyes have seen thy salvation which thou hast prepared in the presence of all peoples, a light for revelation to the Gentiles, and for glory to thy people Israel.
%</"d0589">
%<*"d0590">
Lord, our God, who once betrothed yourself to the Church from the nations as a pure virgin, bless this betrothal, and unite and preserve these servants of yours in peace and concord.
%</"d0590">
%<*"d0591">
Lord, save your people and bless your inheritance, granting to faithful Christians victories over their enemies, and protecting your commonwealth by your Cross.
%</"d0591">
%<*"d0592">
Lord, save your people, and bless your inheritance, granting to faithful Christians victory over their enemies, and guarding your commonwealth by your Cross.
%</"d0592">
%<*"d0593">
Make firm, Master.
%</"d0593">
%<*"d0594">
Make majestic, Master.
%</"d0594">
%<*"d0595">
Make us worthy of your gifts, O Virgin Theotokos, overlooking our errors, and granting healing to those who receive your blessing in faith, O Most Pure one.
%</"d0595">
%<*"d0596">
Maker and Master of the ages, God of all creation, truly beyond being, bless the circle of the year, and in your infinite mercy, compassionate Lord, save all who worship you, the only Master, and who cry in fear, O Redeemer: Grant to all a year of prosperity.
%</"d0596">
%<*"d0597">
March 25th. The Annunciation of the Mother of God.
%</"d0597">
%<*"d0598">
Master, Christ God, King of the ages, and Creator of all things, I thank you for all the good things you have given me, and for Communion in your most pure and life-giving mysteries Therefore I pray you, O Good One, Lover of mankind: guard me under your protection and in the shadow of your wings; and grant that until my last breath I may share worthily and with a pure conscience in your holy things for forgiveness of sins and everlasting life. For you are the Bread of life, the source of sanctification, the giver of blessings; and to you we give glory, with the Father and the Holy Spirit, now and for ever, and to the ages of ages. Amen.
%</"d0598">
%<*"d0599">
Master, bless the herald of the Good Tidings of the Holy Apostle and Evangelist N.
%</"d0599">
%<*"d0600">
Master, bless the holy Bread.
%</"d0600">
%<*"d0601">
Master, bless the holy Cup.
%</"d0601">
%<*"d0602">
Master, bless the incense.
%</"d0602">
%<*"d0603">
Master, bless the sticharion and orarion.
%</"d0603">
%<*"d0604">
Master, bless them both.
%</"d0604">
%<*"d0605">
Master, command.
%</"d0605">
%<*"d0606">
Master, forgive our transgressions. Holy One, visit us and heal our infirmities, for your name's sake.
%</"d0606">
%<*"d0607">
Master, give the blessing.
%</"d0607">
%<*"d0608">
Master, lift up.
%</"d0608">
%<*"d0609">
Master, lover of humankind, Lord Jesus Christ, my God, do not let these holy Mysteries be for my condemnation because I am unworthy, but rather for the cleansing and sanctification of both soul and body and as a pledge of the life and kingdom to come. It is good for me to cleave to God, to place in the Lord the hope of my salvation.
%</"d0609">
%<*"d0610">
May (on Sundays: he who rose from the dead,) Christ, our true God, through the prayers of his most pure and holy Mother, by the power of the precious and life-giving Cross, through the protection of the honoured, Bodiless Powers of heaven, through the intercessions of the honoured, glorious Prophet, Forerunner and Baptist, John, of the holy, glorious and all-praised Apostles, of the holy, glorious and triumphant Martyrs, of our venerable and God-bearing Fathers and Mothers, of Saint N. (of the Church), of our Father among the Saints John Chrysostom, Archbishop of Constantinople, of the holy and righteous forebears of God, Joachim and Anna, of Saint N. whose memory we keep today, and of all the Saints, have mercy on us and save us, for he is good and loves mankind.
%</"d0610">
%<*"d0611">
May (on Sundays: he who rose from the dead,) Christ, our true God, through the prayers of his most pure and holy Mother, of the holy, glorious and all-praised Apostles, of Saint N. (of the Church), of Saint N. whose memory we keep today, of our Father among the Saints John Chrysostom, Archbishop of Constantinople, and of all the Saints, have mercy on us and save us, for he is good and loves mankind.
%</"d0611">
%<*"d0612">
May God, through the prayers of the holy, glorious Apostle and Evangelist N, grant you to proclaim the word with much power, for the fulfilling of the Gospel of his Beloved Son, our Lord Jesus Christ.
%</"d0612">
%<*"d0613">
May He, who as the immortal King, has dominion over the living and the dead and is risen from the dead, Christ our true God, by the prayers of His most holy Mother; of the holy and all-glorious Apostles; of our holy God-bearing Fathers; of the holy and glorious forefathers Abraham, Isaac and Jacob; of his holy and righteous friend Lazarus the four days dead; and of all the saints, assign to the dwelling place of the righteous the soul of his departed servant (\u2026..) who has departed from us, grant it rest in the bosom of Abraham and number it among the righteous, and may he have mercy upon us, for He is good and loving-kind.
%</"d0613">
%<*"d0614">
May He, who is risen from the dead, and by death has overcome death, and to them in the graves has given life, Christ our true God, by the prayers of His most holy Mother; of the holy and all-glorious Apostles; of our holy God-bearing Fathers; and of all the saints, assign to the dwelling place of the righteous the soul of his departed servant (\u2026..) who has departed from us, grant it rest in the bosom of Abraham and number it among the righteous, and may he have mercy upon us, for He is good and loving-kind.
%</"d0614">
%<*"d0615">
May the Lord God remember your diaconate in his Kingdom, always, now and for ever, and to the ages of ages.
%</"d0615">
%<*"d0616">
May the Lord God remember your priesthood in his Kingdom, always, now and for ever, and to the ages of ages.
%</"d0616">
%<*"d0617">
May the Lord bless you from Sion, and may you see the good things of Jerusalem all the days of your life.
%</"d0617">
%<*"d0618">
May the blessing and mercy of the Lord come upon you.
%</"d0618">
%<*"d0619">
May your holy Body, Lord Jesus Christ, our God, bring me eternal life, and your precious Blood forgiveness of sins. May this Eucharist bring me joy, health and gladness; and at your dread Second Coming make me, a sinner, worthy to stand at the right hand of your glory, at the prayers of your all-pure Mother and of all your Saints. Amen.
%</"d0619">
%<*"d0620">
Meanwhile he censes the Sanctuary and, from the Holy Doors, the Icons the People, then re-enters the holy Altar and censes the holy Table once again and then the Priest, and puts the censer in its place.
%</"d0620">
%<*"d0621">
Meanwhile the *Deacon holding the Paten above the Chalice carefully wipes the particles remaining on the Paten into the Chalice.
%</"d0621">
%<*"d0622">
Meanwhile the Priest lifts the Aer and waves it above the Gifts. At the words and ascended into heaven\u2026, he kisses the Cross in the middle of it, folds it and puts it to one side with the veils.
%</"d0622">
%<*"d0623">
Mode 2. Kontakion of the Mother of God.
%</"d0623">
%<*"d0624">
Mode 2. Psalm 102
%</"d0624">
%<*"d0625">
Mode 2. Psalm 145
%</"d0625">
%<*"d0626">
Mode 4.
%</"d0626">
%<*"d0627">
Mode 4. Lifted up on the Cross.
%</"d0627">
%<*"d0628">
Mode 4. Model Melody.
%</"d0628">
%<*"d0629">
Mode 4. Model Melody. (By St Romanos)
%</"d0629">
%<*"d0630">
Mode 4. Today you have appeared.
%</"d0630">
%<*"d0631">
Most holy Mother of God, save us.
%</"d0631">
%<*"d0632">
Most merciful Master, Lord Jesus Christ, our God, through the prayers of our all-pure Lady, Mother of God and Ever-Virgin Mary; by the power of the precious and life-giving Cross; through the protection of the honoured Bodiless Powers of heaven; through the intercessions of the honoured, glorious Prophet, Forerunner and Baptist John; of the holy, glorious and all-praised Apostles; of the holy, glorious and triumphant Martyrs, of our venerable and God-bearing Fathers and Mothers, (the Patron Saint of the Church), of the holy and righteous Forebears of God, Joachim and Anna, of Saint N. whose memory we keep today, and of all your Saints: make our supplication acceptable; grant us forgiveness of our offences; shelter us in the shelter of your wings; drive from us every foe and enemy; make our life peaceful. Lord, have mercy on us and on your world, and save our souls, for you are good and love mankind.
%</"d0632">
%<*"d0633">
Mother of God and Ever-Virgin Mary, with all the Saints, let us entrust ourselves and one another and our whole life to Christ our God.
%</"d0633">
%<*"d0634">
Mother: Amen.
%</"d0634">
%<*"d0635">
Mother: And with your spirit.
%</"d0635">
%<*"d0636">
Mother: Lord, have mercy.
%</"d0636">
%<*"d0637">
Mother: To you, O Lord.
%</"d0637">
%<*"d0638">
Mounted on the throne in heaven, O Christ God, and on the colt on earth, you accepted the praise of the Angels and the hymn of the Children, who cried to you, \u2018Blessed are you, who come to call back Adam'.
%</"d0638">
%<*"d0639">
My soul clung to you; and your right hand upheld me.
%</"d0639">
%<*"d0640">
My soul has longed to desire your judgements at all times. Alleluia.
%</"d0640">
%<*"d0641">
My soul slumbered from weariness; strengthen my by your words. Alleluia.
%</"d0641">
%<*"d0642">
My soul will live and praise you; and your judgements will help me. Alleluia.
%</"d0642">
%<*"d0643">
My soul will rejoice in the Lord, for he has clothed me with a garment of salvation and wrapped me in a robe of gladness; he has placed a crown on my head as on a bridegroom, and adorned me with beauty as a bride.
%</"d0643">
%<*"d0644">
None is holy as you, O Lord my God, who have exalted the horn of your faithful people, O Good one, and established me on the rock of confession of you.
%</"d0644">
%<*"d0645">
Nor tomb nor death overpowered the Mother of God, unsleeping in her prayers, unfailing hope in intercession; for as Mother of Life she has been taken over into life by him who dwelt in her ever-virgin womb.
%</"d0645">
%<*"d0646">
Note
%</"d0646">
%<*"d0647">
November 20th. Forefeast of the Entry of the Mother of God.
%</"d0647">
%<*"d0648">
November 21st. The Entry of the Mother of God into the Temple, and until the Close on the 25th.
%</"d0648">
%<*"d0649">
Now and for ever, and unto the ages of ages. Amen.
%</"d0649">
%<*"d0650">
O God of spirits and all flesh, who trampled down death and crushed the devil, giving life to your world; do you, Lord, give rest to the soul of your servant N., who has fallen asleep, in a place of light, a place of green pasture, a place of refreshment, whence pain, grief and sighing have fled away. Pardon, O God, as you are good and love mankind, every sin committed by him/her in word or deed or thought, because there is none who will live and not sin, for you alone are without sin; your righteousness is an everlasting righteousness, and your word is truth.
%</"d0650">
%<*"d0651">
O God of spirits and all flesh, who trampled down death and crushed the devil, giving life to your world; do you, Lord, give rest to the soul(s) of your servant(s) N. (and N.), who has (have) fallen asleep, in a place of light, a place of green pasture, a place of refreshment, whence pain, grief and sighing have fled away. Pardon, O God, as you are good and love mankind, every sin committed by him/her (them) in word or deed or thought, because there is none who will live and not sin, for you alone are without sin; your righteousness is an everlasting righteousness, and your word is truth.
%</"d0651">
%<*"d0652">
O Holy Martyrs, whose example and whose deeds have won the crown of glory. Do intercede before our Lord, that he may have mercy on our souls evermore.
%</"d0652">
%<*"d0653">
O Lord, all-loving King, who brought all things to completion by your word, and ordered the earth to produce fruits of every kind for our enjoyment and nourishment; who showed Daniel and the three Youths to be healthier through seeds and pulses than those who had fed sumptuously in Babylon, bless these seeds, mixed with different fruits, and make holy those who partake of them. For they have been offered by your servants for your glory, and in honour of Saint N., and in memory of those who have been made perfect in death in the Orthodox faith. Grant also, loving Lord, to those who have prepared these things and who celebrate this memorial, all their requests which are for their salvation and the enjoyment of your eternal blessings; at the prayers of our all-pure Lady, the Mother of God and ever-virgin Mary, of Saint N., whose memory we celebrate, and of all your Saints.
%</"d0653">
%<*"d0654">
O Master, Guide to wisdom, Giver of prudent counsel, Instructor of the foolish and Champion of the poor, make firm my heart, and give it understanding. Word of the Father, give me a word: for see, I shall not stop my lips from crying out to you: In your mercy, have mercy on me who have fallen!
%</"d0654">
%<*"d0655">
O Pure one, full of God's grace, you have given your sacred Robe as a rampart of incorruption to all the faithful; with it you covered your sacred body, O divine protection of mankind. We celebrate its deposition with love, and with shouts we faithfully cry aloud, \u2018Hail Virgin, boast of Christians!'
%</"d0655">
%<*"d0656">
O loving Lord, you have shown today that the tabernacle made with hands is, by your dispensation, the dwelling of your glory which is beyond understanding.
%</"d0656">
%<*"d0657">
O only pure and spotless Virgin, who bore God without seed, intercede for the salvation of the soul of your servant.
%</"d0657">
%<*"d0658">
O only pure and spotless Virgin, who bore God without seed, intercede for the salvation of the soul(s) of your servant(s).
%</"d0658">
%<*"d0659">
O you O Lord, who does watch over children in the present life, and because of their simplicity of mind and innocence, does reward them in the world to come with a place in Abraham's bosom and cause them to dwell in bright and radiant places where the souls of the righteous dwell, do you receive in peace the soul of your servant (.....) for you did say: for of such is the Kingdom of heaven.
%</"d0659">
%<*"d0660">
ODE 1
%</"d0660">
%<*"d0661">
ODE 3
%</"d0661">
%<*"d0662">
ODE 4
%</"d0662">
%<*"d0663">
ODE 5
%</"d0663">
%<*"d0664">
ODE 6
%</"d0664">
%<*"d0665">
ODE 7
%</"d0665">
%<*"d0666">
ODE 8
%</"d0666">
%<*"d0667">
ODE 9
%</"d0667">
%<*"d0668">
OF THE LITTLE CANON OF SUPPLICATION
%</"d0668">
%<*"d0669">
ORDINATION HYMNS
%</"d0669">
%<*"d0670">
Ode 3. None is holy as you.
%</"d0670">
%<*"d0671">
Ode 4. Irmos.
%</"d0671">
%<*"d0672">
Ode 6. Irmos.
%</"d0672">
%<*"d0673">
Ode 6. Watching life's sea.
%</"d0673">
%<*"d0674">
Ode 7. An Angel made the furnace.
%</"d0674">
%<*"d0675">
Ode 7. Irmos.
%</"d0675">
%<*"d0676">
Ode 8. Irmos.
%</"d0676">
%<*"d0677">
Ode 9. Irmos.
%</"d0677">
%<*"d0678">
Ode 9. It is impossible for humans to see God.
%</"d0678">
%<*"d0679">
Of Your mystical supper, O Son of God, accept me this day as a partaker; for I will not speak of the mystery to Your enemies, nor will I give You a kiss like Judas; but like the thief I will acknowledge You: remember me, O Lord, in Your kingdom.
%</"d0679">
%<*"d0680">
Of old you formed me from nothing and honoured me with your divine image, but because I transgressed your commandment, you returned me to the earth from which I was taken; bring me back to your likeness, my ancient beauty.
%</"d0680">
%<*"d0681">
Of your mystical Supper, Son of God, receive me today as a communicant; for I will not tell of the Mystery to your enemies; I will not give you a kiss, like Judas; but like the Thief I confess you: Remember me, Lord, in your Kingdom.
%</"d0681">
%<*"d0682">
On August 14th. Forefeast of the Dormition.
%</"d0682">
%<*"d0683">
On August 15th, the Dormition of the Mother of God, and until the Close on the 24th (or the 28th).
%</"d0683">
%<*"d0684">
On August 1st, the Procession of the life-giving Cross.
%</"d0684">
%<*"d0685">
On August 31st. The Deposition of the Cincture of the Mother of God.
%</"d0685">
%<*"d0686">
On August 5th, Kontakion of the Forefeast of the Transfiguration of the Lord.
%</"d0686">
%<*"d0687">
On August 6th. The Transfiguration of the Lord.
%</"d0687">
%<*"d0688">
On Carnival Sunday.
%</"d0688">
%<*"d0689">
On Cheese Sunday.
%</"d0689">
%<*"d0690">
On February 1st, the Forefeast of the Meeting.
%</"d0690">
%<*"d0691">
On July 25th, the Dormition of Saint Anne, mother of the Mother of God.
%</"d0691">
%<*"d0692">
On July 2nd, the Deposition of the precious Robe of our most holy Lady the Mother of God in Blachernae.
%</"d0692">
%<*"d0693">
On Lazarus Saturday and Palm Sunday.
%</"d0693">
%<*"d0694">
On March 24th, the Eve of the Annunciation.
%</"d0694">
%<*"d0695">
On March 25th the Annunciation of the Mother of God.
%</"d0695">
%<*"d0696">
On Palm Sunday.
%</"d0696">
%<*"d0697">
On Saturday evening.
%</"d0697">
%<*"d0698">
On Sundays the following Antiphon is usually sung:
%</"d0698">
%<*"d0699">
On Sundays:
%</"d0699">
%<*"d0700">
On feasts of the Lord and their leave-taking the Apolytikion of the Feast is sung; during Eastertide Christ has risen from the dead...
%</"d0700">
%<*"d0701">
On other days as appointed in the Typikon.
%</"d0701">
%<*"d0702">
On the 1st Sunday of Lent, the Sunday of Orthodoxy.
%</"d0702">
%<*"d0703">
On the 2nd Sunday of Lent, of St Gregory Palamas.
%</"d0703">
%<*"d0704">
On the 3rd Sunday of Lent, of the Veneration of the Cross.
%</"d0704">
%<*"d0705">
On the 4th Sunday of Lent, of St John of the Ladder.
%</"d0705">
%<*"d0706">
On the 5th Sunday of Lent, of St Mary of Egypt.
%</"d0706">
%<*"d0707">
On the 5th Sunday of the Fast, if it falls after the Annunciation. (Except in churches dedicated to the Mother of God, which should use the appropriate Kontakion.)
%</"d0707">
%<*"d0708">
On the Ascension of our Lord, and until the Leave-taking.
%</"d0708">
%<*"d0709">
On the Holy and Great Sunday of Pascha.
%</"d0709">
%<*"d0710">
On the Saturday of Carnival (of the Departed).
%</"d0710">
%<*"d0711">
On the Saturday of the holy and righteous Lazarus.
%</"d0711">
%<*"d0712">
On the Saturdays for the Departed the petitions in the Litany are as follows:
%</"d0712">
%<*"d0713">
On the Sunday of All Saints.
%</"d0713">
%<*"d0714">
On the Sunday of Pentecost, and until the Leave-taking.
%</"d0714">
%<*"d0715">
On the Sunday of the Holy Apostle Thomas.
%</"d0715">
%<*"d0716">
On the Sunday of the Holy Fathers.
%</"d0716">
%<*"d0717">
On the Sunday of the Myrrhbearers.
%</"d0717">
%<*"d0718">
On the Sunday of the Prodigal Son.
%</"d0718">
%<*"d0719">
On the Sunday of the Tax-collector and the Pharisee.
%</"d0719">
%<*"d0720">
On the Thursday of the Ascension, and until the Leave-taking.
%</"d0720">
%<*"d0721">
On the Wednesday of Mid-Pentecost and until the Leave- taking.
%</"d0721">
%<*"d0722">
On the Wednesday of Mid-Pentecost, and until the Leave-taking.
%</"d0722">
%<*"d0723">
On the holy and great Sunday of Pascha, and until the Close, and on Thomas Sunday.
%</"d0723">
%<*"d0724">
On weekdays, outside Eastertide or a feast of the Lord, we sing each time:
%</"d0724">
%<*"d0725">
On your glorious memorial the whole inhabited world, mystically adorned by God's Spirit, cries out to you with joy: Hail, Virgin, boast of Christians.
%</"d0725">
%<*"d0726">
One of the soldiers pierced his side with a lance, and at once there came out blood and water; and he who saw it has borne witness, and his witness is true.
%</"d0726">
%<*"d0727">
Only-begotten Son and Word of God, * who, being immortal, * accepted for our salvation * to take flesh from the holy Mother of God * and Ever-Virgin Mary, and without change became man; * you were crucified, Christ God, * by death trampling on death, * though being one of the Holy Trinity, * glorified with the Father and the Holy Spirit: * save us!
%</"d0727">
%<*"d0728">
Open the gate of compassion to us, blessed Mother of God; hoping in you, may we not fail. Through you may we be delivered from adversities, for you are the salvation of the Christian race.
%</"d0728">
%<*"d0729">
Or, according to some ancient Orders:
%</"d0729">
%<*"d0730">
Orthodoxy's beacon, support and teacher of the Church, fair glory of monastics, invincible champion of theologians, Wonderworker Gregory, the boast of Thessalonika and preacher of grace, intercede without ceasing that our souls may be saved.
%</"d0730">
%<*"d0731">
Other Troparia.
%</"d0731">
%<*"d0732">
Our Father in heaven, may your name be hallowed; your kingdom come; your will be done on earth as in heaven. Give us today our daily bread, and forgive us our debts, as we forgive our debtors. And do not lead us into temptation, but deliver us from the evil one
%</"d0732">
%<*"d0733">
Our Father who art in heaven, hallowed be thy name. Thy kingdom come, thy will be done, on earth, as it is in heaven. Give us this day our daily bread, and forgive us our trespasses as we forgive those who trespass against us, and lead us not into temptation, but deliver us from the evil one.
%</"d0733">
%<*"d0734">
Our Father, in heaven, may your name be hallowed, your kingdom come, your will be done on earth as in heaven. Give us today our daily bread, and forgive us our debts, as we forgive our debtors, and do not lead us into temptation, but deliver us from the evil one.
%</"d0734">
%<*"d0735">
Our Father, in heaven, may your name be hallowed, your kingdom come; your will be done on earth as in heaven. Give us today our daily bread, and forgive us our debts, as we forgive our debtors, and do not lead us into temptation, but deliver us from the evil one.
%</"d0735">
%<*"d0736">
Our Father, in heaven, may your name be hallowed; your kingdom come; your will be done on earth as in heaven. Give us today our daily bread, and forgive us our debts, as we forgive our debtors. And do not lead us into temptation, but deliver us from the evil one.
%</"d0736">
%<*"d0737">
Our Father, in heaven, may your name be sanctified, your kingdom come; your will be done on earth as in heaven. Give us today our daily bread, and forgive us our debts, as we forgive our debtors, and do not lead us into temptation, but deliver us from the evil one.
%</"d0737">
%<*"d0738">
PENITENT
%</"d0738">
%<*"d0739">
PEOPLE
%</"d0739">
%<*"d0740">
PEOPLE we praise you, we bless you, we give thanks to you, O Lord, and we pray to you, our God.
%</"d0740">
%<*"d0741">
PRAYER BEHIND THE AMBO
%</"d0741">
%<*"d0742">
PRAYER FOR THE CATECHUMENS
%</"d0742">
%<*"d0743">
PRAYER OF FORGIVENESS
%</"d0743">
%<*"d0744">
PRAYER OF SUPPLICATION
%</"d0744">
%<*"d0745">
PRAYER OF THE CHERUBIC HYMN
%</"d0745">
%<*"d0746">
PRAYER OF THE ENTRANCE
%</"d0746">
%<*"d0747">
PRAYER OF THE FIRST ANTIPHON
%</"d0747">
%<*"d0748">
PRAYER OF THE GOSPEL
%</"d0748">
%<*"d0749">
PRAYER OF THE OFFERTORY
%</"d0749">
%<*"d0750">
PRAYER OF THE SECOND ANTIPHON
%</"d0750">
%<*"d0751">
PRAYER OF THE THIRD ANTIPHON
%</"d0751">
%<*"d0752">
PRAYER OF THE TRISAGION
%</"d0752">
%<*"d0753">
PRAYERS BEFORE HOLY COMMUNION
%</"d0753">
%<*"d0754">
PRAYERS FOR VARIOUS OCCASIONS
%</"d0754">
%<*"d0755">
PRELIMINARIES TO THE LITURGY
%</"d0755">
%<*"d0756">
PRIE/DEA: Axios! (x3)
%</"d0756">
%<*"d0757">
PRIEST
%</"d0757">
%<*"d0758">
PRIEST he who has authority over the living and the dead, as immortal King, and who rose from the dead, Christ, our true God, through the prayers of his most pure and holy Mother, of the holy, glorious and all-praised Apostles, of our venerable and God-bearing fathers, of the holy and glorious forefathers, Abraham, Isaac and Jacob, of the holy and righteous Lazarus, for four days dead, the friend of Christ, and of all the Saints, establish in the tents of the righteous the soul(s) of his servant(s) who has (have) gone from us, give him/her (them) rest in the bosom of Abraham, and number him/her (them) with the righteous; and have mercy on us and save us, for he is good and loves mankind.
%</"d0758">
%<*"d0759">
PRIEST offering you your own from your own \u2014 in all things and for all things \u2014
%</"d0759">
%<*"d0760">
Pardon and forgiveness of our sins and offences, let us ask of the Lord.
%</"d0760">
%<*"d0761">
Patient: Amen.
%</"d0761">
%<*"d0762">
Patient: Lord, have mercy.
%</"d0762">
%<*"d0763">
Patient: Lord, have mercy. (Repeated after each petition)
%</"d0763">
%<*"d0764">
Peace be to you all
%</"d0764">
%<*"d0765">
Peace be with all.
%</"d0765">
%<*"d0766">
Peace be with you.
%</"d0766">
%<*"d0767">
Peace to all.
%</"d0767">
%<*"d0768">
Peace to you the reader
%</"d0768">
%<*"d0769">
Peace to you the reader.
%</"d0769">
%<*"d0770">
Peace to you.
%</"d0770">
%<*"d0771">
Peace unto all.
%</"d0771">
%<*"d0772">
Peace unto you.
%</"d0772">
%<*"d0773">
People
%</"d0773">
%<*"d0774">
Pierce, Master.
%</"d0774">
%<*"d0775">
Plagal of the Second Tone
%</"d0775">
%<*"d0776">
Pray for me, brother and fellow celebrant.
%</"d0776">
%<*"d0777">
Prayer at the Bowing of Heads
%</"d0777">
%<*"d0778">
Prayer of the Breaking of Bread
%</"d0778">
%<*"d0779">
Priest
%</"d0779">
%<*"d0780">
Priest (aloud):
%</"d0780">
%<*"d0781">
Priest, blessing both the Chalice and Paten, says:
%</"d0781">
%<*"d0782">
Priest, blessing the People:
%</"d0782">
%<*"d0783">
Priest, bowing his head with great compunction, says:
%</"d0783">
%<*"d0784">
Priest, raising his hands:
%</"d0784">
%<*"d0785">
Prokeimenon of the Apostle. Mode\u2026 Psalm of David.
%</"d0785">
%<*"d0786">
Prokeimenon, Tone 3.
%</"d0786">
%<*"d0787">
Protection of Christians that cannot be put to shame, unfailing mediation with the Maker, do not despise the voices of us sinners as we pray; but, in your love, be quick to help us who cry to you with faith: Hasten to intercede, make speed to entreat, O Mother of God, for you ever protect those who honour you.
%</"d0787">
%<*"d0788">
Psalm 102
%</"d0788">
%<*"d0789">
Psalm 142
%</"d0789">
%<*"d0790">
Psalm 148
%</"d0790">
%<*"d0791">
Psalm 149
%</"d0791">
%<*"d0792">
Psalm 150
%</"d0792">
%<*"d0793">
Psalm 19
%</"d0793">
%<*"d0794">
Psalm 20
%</"d0794">
%<*"d0795">
Psalm 3
%</"d0795">
%<*"d0796">
Psalm 37
%</"d0796">
%<*"d0797">
Psalm 50
%</"d0797">
%<*"d0798">
Psalm 62
%</"d0798">
%<*"d0799">
Psalm 87
%</"d0799">
%<*"d0800">
READER
%</"d0800">
%<*"d0801">
Reader
%</"d0801">
%<*"d0802">
Reader, coming into the middle of the church, says:
%</"d0802">
%<*"d0803">
Receive our prayer, * you who sit at the right hand of the Father, * and have mercy on us.
%</"d0803">
%<*"d0804">
Receive the Body of Christ; taste from the immortal fount. Alleluia, Alleluia, Alleluia.
%</"d0804">
%<*"d0805">
Remember also, Lord, me your unworthy servant, and pardon me every offence, both voluntary and involuntary.
%</"d0805">
%<*"d0806">
Remember, Lord, N.
%</"d0806">
%<*"d0807">
Remember, Master, Lover of mankind, every bishopric of the Orthodox, our Archbishop N., the honoured order of presbyters, the diaconate in Christ and every order of clergy, our brothers and fellow celebrants, priests and deacons, and the whole priestly and monastic order.
%</"d0807">
%<*"d0808">
Remembering our most holy, pure, blessed, glorious Lady Theotokos and ever-Virgin Mary, with all the saints, let us commit ourselves and one another and our whole life to Christ our God.
%</"d0808">
%<*"d0809">
Rulers have persecuted me for no reason; and my heart has been in awe of your words. Alleluia.
%</"d0809">
%<*"d0810">
SECOND ANTIPHON
%</"d0810">
%<*"d0811">
SECOND PRAYER OF THE FAITHFUL
%</"d0811">
%<*"d0812">
SERVICE
%</"d0812">
%<*"d0813">
SERVICE OF THE ARTOKLASIA
%</"d0813">
%<*"d0814">
SERVICE OF THE TRISAGION FOR THE DEPARTED
%</"d0814">
%<*"d0815">
SHORT LITANY
%</"d0815">
%<*"d0816">
Sacrifice, Master.
%</"d0816">
%<*"d0817">
Save your servants, <br> from every danger, O Mother of God, <br> for next after God <br> we all fly for refuge to you <br> as unbreachable wall and protection.
%</"d0817">
%<*"d0818">
Second Prayer.
%</"d0818">
%<*"d0819">
See, to divine Communion I approach; My Maker, burn me not as I partake, For you are fire consuming the unworthy, But rather cleanse me now of every stain.
%</"d0819">
%<*"d0820">
September 13th. Dedication of the church of the Resurrection.
%</"d0820">
%<*"d0821">
September 14th. The Exaltation of the Precious Cross, and until the Leave-taking on the 21st.
%</"d0821">
%<*"d0822">
September 14th. The Universal Exaltation of the Precious Cross, and until the Close on the 21st.
%</"d0822">
%<*"d0823">
September 1st. Beginning of the Indiction.
%</"d0823">
%<*"d0824">
September 7th. Forefeast of the Nativity of the Mother of God.
%</"d0824">
%<*"d0825">
September 8th. The Nativity of the Mother of God, and until the Close on the 12th.
%</"d0825">
%<*"d0826">
September 8th. The Nativity of the Mother of God, and until the Leave-taking on the 12th.
%</"d0826">
%<*"d0827">
Shelter us in the shelter of your wings; drive away from us e\u00ADvery enemy and adversary. Bring peace, O Lord, to our lives, have mercy on us and on your world, and save our souls, for you are good and the lover of mankind.
%</"d0827">
%<*"d0828">
Shelter, Master.
%</"d0828">
%<*"d0829">
Singers
%</"d0829">
%<*"d0830">
Son of God, risen from the dead, save us who sing to you: Alleluia!
%</"d0830">
%<*"d0831">
Son of God, wonderful in the Saints, save us who sing to you: Alleluia!
%</"d0831">
%<*"d0832">
Sprinkle me with hyssop and I shall be cleansed, wash me and I shall be whiter than snow.
%</"d0832">
%<*"d0833">
Stretch out your hand, Lord, from your dwelling on high, and strengthen me for your service which now awaits me, so that, standing uncondemned before your dread altar, I may offer the sacrifice without shedding of blood. For yours is the power and the glory to the ages of ages. Amen.
%</"d0833">
%<*"d0834">
THANKSGIVING AFTER HOLY COMMUNION
%</"d0834">
%<*"d0835">
THANKSGIVING AND DISMISSAL
%</"d0835">
%<*"d0836">
THE CREED
%</"d0836">
%<*"d0837">
THE DIVINE LITURGY OF OUR FATHER AMONG THE SAINTS JOHN CHRYSOSTOM
%</"d0837">
%<*"d0838">
THE END OF THE DIVINE LITURGY OF JOHN CHRYSOSTOM
%</"d0838">
%<*"d0839">
THE EVLOGITARIA FOR THE DEPARTED
%</"d0839">
%<*"d0840">
THE GREAT DOXOLOGY
%</"d0840">
%<*"d0841">
THE HOLY GOSPEL
%</"d0841">
%<*"d0842">
THE HOMILY
%</"d0842">
%<*"d0843">
THE LITURGY OF THE CATECHUMENS
%</"d0843">
%<*"d0844">
THE LITURGY OF THE FAITHFUL
%</"d0844">
%<*"d0845">
THE LORD'S PRAYER
%</"d0845">
%<*"d0846">
THE PREPARATION FOR HOLY COMMUNION
%</"d0846">
%<*"d0847">
THE READINGS FROM THE NEW TESTAMENT
%</"d0847">
%<*"d0848">
THE SERVICE OF PREPARATION
%</"d0848">
%<*"d0849">
THIRD ANTIPHON
%</"d0849">
%<*"d0850">
TO THE MOST HOLY MOTHER OF GOD
%</"d0850">
%<*"d0851">
TONE 8
%</"d0851">
%<*"d0852">
Taking another prosphora, he says:
%</"d0852">
%<*"d0853">
Taking the holy Gospel, he makes the sign of the Cross with it over the folded Antimension and says aloud:
%</"d0853">
%<*"d0854">
That the Lord our God may establish his/her soul where the righteous rest.
%</"d0854">
%<*"d0855">
That the Lord our God may establish their souls where the righteous rest.
%</"d0855">
%<*"d0856">
That the whole day may be perfect, holy, peaceful and sinless, let us ask of the Lord.
%</"d0856">
%<*"d0857">
That this water may be sanctified by the power, energy and descent of the Holy Spirit, let us pray to the Lord.
%</"d0857">
%<*"d0858">
That we may live out the rest of our days in peace and repentance, let us ask of the Lord.
%</"d0858">
%<*"d0859">
The *Deacon chants the Gospel for the day. When it is finished the Priest blesses the Deacon saying:
%</"d0859">
%<*"d0860">
The *Deacon comes out through the Holy Doors, goes to his usual place and says:
%</"d0860">
%<*"d0861">
The *Deacon crosses his hands and elevates the Chalice and Paten as the Priest says aloud:
%</"d0861">
%<*"d0862">
The *Deacon divides the two remaining parts of the Lamb (NI and KA) into small pieces and places them in the Chalice, which he covers with the Communion cloth and lays the Spoon on top.
%</"d0862">
%<*"d0863">
The *Deacon pours the hot water into the Chalice in the form of a cross, saying:
%</"d0863">
%<*"d0864">
The *Deacon takes the Star from the Paten, making the sign of the Cross with it over the Paten, kisses it and lays it aside on the Holy Table.
%</"d0864">
%<*"d0865">
The *Deacon, standing at the Holy Door, recites the Diptychs of the living and then exclaims:
%</"d0865">
%<*"d0866">
The Angel standing by the grave cried to the women bearing myrrh: Myrrh is fitting for the dead, but Christ has shown himself a stranger to corruption. But cry aloud: The Lord has risen, granting the world his great mercy.
%</"d0866">
%<*"d0867">
The Apolytikia for the day and for the dedication of the Church are sung and after
%</"d0867">
%<*"d0868">
The Bishop reads the Gospel.
%</"d0868">
%<*"d0869">
The Bishop-Elect continues as follows:
%</"d0869">
%<*"d0870">
The Body and Blood of Christ, for the forgiveness of sins and everlasting life.
%</"d0870">
%<*"d0871">
The Chanters repeat this twice.
%</"d0871">
%<*"d0872">
The Church from the nations, does not have sand but Christ as her foundation, is crowned with the unapproachable beauty, and adorned with the diadem of the Kingdom.
%</"d0872">
%<*"d0873">
The Church has been declared a heaven filled with light, which guides all the faithful to the light; standing in it we cry: Establish this house, O Lord.
%</"d0873">
%<*"d0874">
The Church, which has you as her unshaken foundation, O Christ, is today crowned with the Cross as with a royal diadem.
%</"d0874">
%<*"d0875">
The Deacon again goes and stands in front of the icon of the Mother of God, and after the completion of the Antiphon he comes and stands in his usual place, bows and says the
%</"d0875">
%<*"d0876">
The Deacon bows his head and says to the priest in a low voice:
%</"d0876">
%<*"d0877">
The Deacon bows his head to the Priest, and says to the Priest:
%</"d0877">
%<*"d0878">
The Deacon comes out and stands in his usual place.
%</"d0878">
%<*"d0879">
The Deacon enters the Sanctuary by the south door.
%</"d0879">
%<*"d0880">
The Deacon enters the Sanctuary, ties his orarion in the form of a Cross and standing on the right of the Priest says:
%</"d0880">
%<*"d0881">
The Deacon hands the Chalice to the Priest, who gives Communion to the People, saying to each communicant:
%</"d0881">
%<*"d0882">
The Deacon kisses the Priest's hand, goes to the back of the Holy Table, and communicates like the Priest.
%</"d0882">
%<*"d0883">
The Deacon kisses the Priest's right hand and goes to vest.
%</"d0883">
%<*"d0884">
The Deacon points to the Chalice with his Orarion and says:
%</"d0884">
%<*"d0885">
The Deacon pours sufficient wine and water into the Chalice, saying first to the Priest:
%</"d0885">
%<*"d0886">
The Deacon re-enters the Sanctuary.
%</"d0886">
%<*"d0887">
The Deacon says to the Priest:
%</"d0887">
%<*"d0888">
The Deacon takes the censer and puts incense in it, saying to the Priest:
%</"d0888">
%<*"d0889">
The Deacon then approaches the Priest, and says:
%</"d0889">
%<*"d0890">
The Deacon, as he approaches, says:
%</"d0890">
%<*"d0891">
The Deacon, having carefully wiped his hand over the Paten with the Sponge, approaches, saying:
%</"d0891">
%<*"d0892">
The Deacon, if there is one, if not, the Priest himself, goes to the Prothesis and prepares the sacred Vessels, placing the Paten on the left and the Chalice on the right. The Priest, wearing all his priestly vestments, goes to the Prothesis.
%</"d0892">
%<*"d0893">
The Deacon, in a low voice, says:
%</"d0893">
%<*"d0894">
The Deacon, stands to the right in front of the Holy Table and says to the Priest as he enters:
%</"d0894">
%<*"d0895">
The Dismissal.
%</"d0895">
%<*"d0896">
The Holy Doors are opened and the Priest hands the Chalice to the *Deacon, who comes out through the Holy Doors, holding the Chalice, and says:
%</"d0896">
%<*"d0897">
The Ikos.
%</"d0897">
%<*"d0898">
The Irmos.
%</"d0898">
%<*"d0899">
The Kontakion, Ikos and Synaxarion for the day are read.
%</"d0899">
%<*"d0900">
The Lamb of God, who takes away the sin of the world, is sacrificed for the life and salvation of the world.
%</"d0900">
%<*"d0901">
The Lord is God, and has appeared to us. Blessed is he who comes in the name of the Lord.
%</"d0901">
%<*"d0902">
The Lord is God, and has appeared to us. Blessed is he who comes in the name of the Lord. (This is sung after each of the following verses by the two choirs alternately)
%</"d0902">
%<*"d0903">
The Lord is King: he has clothed himself with glory.
%</"d0903">
%<*"d0904">
The Lord of all things undergoes circumcision, and in his love cuts off the failings of mortals; today he gives salvation to the world. While in the highest there rejoices the Hierarch of the Creator and bearer of light, Basil, the god-like initiate of Christ.
%</"d0904">
%<*"d0905">
The Lord said to the Jews who had come to him, \u2018Amen, Amen I say to you, that one who hears my word and believes in him who sent me has eternal life; and is not coming to judgement, but has passed over from death to life. Amen, Amen I say to you that the hour is coming and is now, when the dead will hear the voice of the Son of God; and those who have heard will live. For just as the Father has life in himself, so he has given to Son to have life in himself. And he has given him authority to deliver judgement also, because he is son of man. Do not marvel at this; because the hour is coming in which all who are in the tombs will hear his voice; and they will come out, those who have done good to the resurrection of life, but those who have done ill to the resurrection of judgement. I can do nothing of myself; as I hear, I judge, and my judgement is just; because I do not seek my will, but the will of the Father who sent me.'
%</"d0905">
%<*"d0906">
The People sing the Entrance Chant as follows:
%</"d0906">
%<*"d0907">
The Priest and Deacon each takes his sticharion, faces east and makes three bows, saying each time:
%</"d0907">
%<*"d0908">
The Priest and Deacon enter the Sanctuary.
%</"d0908">
%<*"d0909">
The Priest and the Deacon also say the Trisagion, making three bows before the holy Table.
%</"d0909">
%<*"d0910">
The Priest asks the forgiveness of those in the Sanctuary and the rest of the church. Then he approaches the Holy Table and says:
%</"d0910">
%<*"d0911">
The Priest blesses him, saying:
%</"d0911">
%<*"d0912">
The Priest blesses it, saying:
%</"d0912">
%<*"d0913">
The Priest blesses the Chalice, saying:
%</"d0913">
%<*"d0914">
The Priest blesses the People with his hand, saying:
%</"d0914">
%<*"d0915">
The Priest blesses the incense and begins to cense the sanctuary and the whole church as usual.
%</"d0915">
%<*"d0916">
The Priest blesses the incense, saying:
%</"d0916">
%<*"d0917">
The Priest blesses them saying:
%</"d0917">
%<*"d0918">
The Priest blesses them with the words:
%</"d0918">
%<*"d0919">
The Priest bows three times and kisses the Aer over the Gifts, saying in a low voice:
%</"d0919">
%<*"d0920">
The Priest bows, takes the Chalice and says in a low voice:
%</"d0920">
%<*"d0921">
The Priest censes the Chalice three times, saying each time:
%</"d0921">
%<*"d0922">
The Priest censes the Star, the Deacon holding the censer, and places it on the Paten above the holy bread, saying:
%</"d0922">
%<*"d0923">
The Priest censes the first veil, in the same way as the Star, and places it over the Paten, saying:
%</"d0923">
%<*"d0924">
The Priest censes the large veil (the Aer), as before, and with it covers both the Chalice and the Paten, saying:
%</"d0924">
%<*"d0925">
The Priest comes out through the Holy Doors and reads aloud:
%</"d0925">
%<*"d0926">
The Priest cuts it crosswise on the underside, but taking care to leave the seal intact, and says:
%</"d0926">
%<*"d0927">
The Priest distributes the Antidoron, saying to each recipient:
%</"d0927">
%<*"d0928">
The Priest divides the Lamb into four parts, saying:
%</"d0928">
%<*"d0929">
The Priest elevates the Holy Bread and says aloud:
%</"d0929">
%<*"d0930">
The Priest gives the Deacon a portion of the Holy Bread from the part stamped with XC and says:
%</"d0930">
%<*"d0931">
The Priest gives the Dismissal as follows:
%</"d0931">
%<*"d0932">
The Priest gives the censer to the Deacon, who censes around the Holy Table and, in a low voice, remembers to himself those whom he wishes of the living and the dead, while the Priest continues:
%</"d0932">
%<*"d0933">
The Priest kisses the Gospel. The *Deacon, standing in the middle of the church in front of the Priest and raising the sacred Gospel, says aloud:
%</"d0933">
%<*"d0934">
The Priest lifts the Aer and places it on the shoulders of the Deacon, saying:
%</"d0934">
%<*"d0935">
The Priest pierces the seal with the Lance on the right side, immediately below the letters IC, and says:
%</"d0935">
%<*"d0936">
The Priest places the Chalice on the Prothesis, returns, and folds up the Antimension, after making sure that no crumb remains.
%</"d0936">
%<*"d0937">
The Priest prays:
%</"d0937">
%<*"d0938">
The Priest recites the Trisagion for the departed during Bright Week as in the home.
%</"d0938">
%<*"d0939">
The Priest returns to the Sanctuary through the Holy Doors, goes to the table of the Prothesis and says, in a low voice:
%</"d0939">
%<*"d0940">
The Priest says the following Prayer aloud:
%</"d0940">
%<*"d0941">
The Priest says this Prayer:
%</"d0941">
%<*"d0942">
The Priest stands upright and blesses the holy Bread, saying in a low voice:
%</"d0942">
%<*"d0943">
The Priest takes one of the loaves and makes the sign of the Cross over it and says the following prayer aloud:
%</"d0943">
%<*"d0944">
The Priest takes the Gospel from the Deacon, kisses it and blesses the people with it. He then replaces it on the Holy Table.
%</"d0944">
%<*"d0945">
The Priest takes the portion of the Lamb stamped with the letters IC and makes the sign of the Cross with it above the holy Chalice and places it in it, saying:
%</"d0945">
%<*"d0946">
The Priest who is going to celebrate the divine Mystery must be reconciled beforehand with everyone, and have nothing against anyone; he must guard his heart, as far he can, from wicked thoughts; from the evening before he should remain abstinent, and be vigilant until the moment of the divine service.
%</"d0946">
%<*"d0947">
The Priest, blessing the People, says:
%</"d0947">
%<*"d0948">
The Priest, bowing profoundly, continues:
%</"d0948">
%<*"d0949">
The Priest, coming out through the Holy Doors and blessing the People, says:
%</"d0949">
%<*"d0950">
The Priest, lifting up the book of the Gospel, and making the sign of the Cross with it over the Antimension, says in a clear voice:
%</"d0950">
%<*"d0951">
The Priest, placing his right hand on his head, blesses him, saying:
%</"d0951">
%<*"d0952">
The Priest, standing in the holy Doors and facing the People, gives the Great Dismissal as follows:
%</"d0952">
%<*"d0953">
The Priest, while censing, chants:
%</"d0953">
%<*"d0954">
The Reader begins the following Psalms:
%</"d0954">
%<*"d0955">
The Reader reads the Apostle.
%</"d0955">
%<*"d0956">
The Reader reads the Prokeimenon with its verse.
%</"d0956">
%<*"d0957">
The Reader reads the title of the Apostle.
%</"d0957">
%<*"d0958">
The Reader says the following Troparia:
%</"d0958">
%<*"d0959">
The Reading is from the First Epistle of Paul to the Thessalonians. [4:13-18]
%</"d0959">
%<*"d0960">
The Reading is from the Prophecy of Isaias.
%</"d0960">
%<*"d0961">
The Singers begin the Cherubic Hymn to a slow and solemn melody, in the dominant Tone of the day:
%</"d0961">
%<*"d0962">
The Singers continue with the Communion Chant.
%</"d0962">
%<*"d0963">
The Sunday before Christ's Nativity (December 18th\u201324th). Resurrection Apolytikion and of the Forefathers.
%</"d0963">
%<*"d0964">
The Sunday of the holy Forefathers (December 11th\u201317th). Resurrection Apolytikion and of the Forebears.
%</"d0964">
%<*"d0965">
The Word without beginning with Father and with Spirit, born from a Virgin for our salvation, let us believers praise and let us worship him; for he was well pleased to ascend the Cross in the flesh and undergo death, and to raise those who had died, by his glorious Resurrection.
%</"d0965">
%<*"d0966">
The blessing of the Lord and his mercy be upon you, by his grace and love for mankind, always, now and for ever, and to the ages of ages.
%</"d0966">
%<*"d0967">
The choir of Saints has found the source of life and the door of Paradise; may I too find the way through repentance; I am the lost sheep, call me back, O Saviour, and save me.
%</"d0967">
%<*"d0968">
The earth is the Lord's and the fullness thereof, the world and those who dwell therein, you art earth and to earth shall you return.
%</"d0968">
%<*"d0969">
The grace which shone from your mouth like a torch of flame enlightened the whole earth; it laid up for the world the treasures of freedom from avarice; it showed us the height of humility. But as you train us by your words, Father John Chrysostom, intercede with Christ God, the Word, that our souls may be saved.
%</"d0969">
%<*"d0970">
The just is remembered with praises; but for you, O Forerunner, the Lord's testimony suffices. For you were revealed as more praiseworthy than the Prophets, because you were found worthy to baptise in running streams the One they had proclaimed. Therefore you struggled bravely for the truth with joy, and preached to those in Hell a God who had appeared in flesh, who takes away the sin of the world and grants us his great mercy.
%</"d0970">
%<*"d0971">
The mercies of God, the kingdom of heaven and the forgiveness of his/her sins, let us ask of Christ, our immortal King and God.
%</"d0971">
%<*"d0972">
The mercies of God, the kingdom of heaven and the forgiveness of their sins, let us ask of Christ, our immortal King and God.
%</"d0972">
%<*"d0973">
The most pure Temple of the Saviour, the precious Bridal Chamber and Virgin, the sacred Treasury of the glory of God, is being brought today into the house of the Lord; and with her she brings the grace of the divine Spirit; of her God's Angels sing in praise. She is indeed the heavenly Tabernacle.
%</"d0973">
%<*"d0974">
The noble Joseph, taking down your most pure Body from the Tree, wrapped it in a clean shroud with sweet spices and laid it for burial in a new grave. But on the third day you arose, O Lord, granting the world your great mercy.
%</"d0974">
%<*"d0975">
The noble Joseph, taking your most pure Body from the Tree, wrapped it in pure linen with sweet spices and laid it for burial in a new grave.
%</"d0975">
%<*"d0976">
The reading is from the Epistle of St. Paul to the Romans. 6: 9-11
%</"d0976">
%<*"d0977">
The reading is from the Holy Gospel according to N.
%</"d0977">
%<*"d0978">
The reading is from the Holy Gospel according to St. Luke. 18:15-17, 26-27
%</"d0978">
%<*"d0979">
The same Mode.
%</"d0979">
%<*"d0980">
The same tone. By Anatolios.
%</"d0980">
%<*"d0981">
The second Psalm of the Typika or, on Sundays, the following Antiphon:
%</"d0981">
%<*"d0982">
The servant of God N. betroths himself to the servant of God M., in the name of the Father and of the Son and of the Holy Spirit. Amen.
%</"d0982">
%<*"d0983">
Then again the following verses:
%</"d0983">
%<*"d0984">
Then he commemorates the Bishop who ordained him, if he is still living, and then those of the living whose names he has, taking particles for each and saying:
%</"d0984">
%<*"d0985">
Then he cuts particles for the departed, placing them also below the Lamb, saying:
%</"d0985">
%<*"d0986">
Then he gives the Paten, with the covers and the Star, to the Deacon, who shows them to the People and then takes them to the table of the Prothesis, going round behind the Holy Table. He then unties his orarion.
%</"d0986">
%<*"d0987">
Then he says:
%</"d0987">
%<*"d0988">
Then he takes an eighth particle and places it below the seventh, saying:
%</"d0988">
%<*"d0989">
Then he takes the covered Paten and places it with every care and reverence on the Deacon's head, while he himself takes the Holy Chalice, likewise covered.
%</"d0989">
%<*"d0990">
Then he takes the covers from the sacred Paten and the Holy Chalice and lays them to one side on the Holy Table. He takes the Aer from the shoulders of the Deacon, holds it over the censer and lays it over the Chalice and Paten. Then he takes the censer and censes the Gifts three times, as the Deacon says:
%</"d0990">
%<*"d0991">
Then he, pointing to them both, says:
%</"d0991">
%<*"d0992">
Then taking some earth with the shovel he spreads it on the body crosswise, saying:
%</"d0992">
%<*"d0993">
Then the Beatitudes, or the Third Antiphon, are sung. On Sundays the following Antiphon in the Mode of the week. Psalm 117
%</"d0993">
%<*"d0994">
Then the Best Man changes the rings.
%</"d0994">
%<*"d0995">
Then the Deacon says to the Priest:
%</"d0995">
%<*"d0996">
Then the Deacon, followed by the Priest, enters the Sanctuary through the Holy Doors and places the Gospel on the Holy Table.
%</"d0996">
%<*"d0997">
Then the Deacon, pointing to the Chalice with his orarion, says in a low voice:
%</"d0997">
%<*"d0998">
Then the Deacon, pointing to the holy Bread with his orarion, says in a low voice:
%</"d0998">
%<*"d0999">
Then the Deacon, standing in his usual place, says the following
%</"d0999">
%<*"d1000">
Then the Litany for the Catechumens, those preparing for Baptism.
%</"d1000">
%<*"d1001">
Then the Litany of supplication.
%</"d1001">
%<*"d1002">
Then the Preacher instructs the people in the word of God.
%</"d1002">
%<*"d1003">
Then the Priest and Deacon, having kissed one of the loaves, re-enter the Sanctuary, chanting:
%</"d1003">
%<*"d1004">
Then the Priest blesses the cross on the back of his sticharion, kisses it and says:
%</"d1004">
%<*"d1005">
Then the Priest kisses the holy Gospel and the Deacon the Holy Table and the Priest's hand. He then goes out by the north door and stands in his usual place in front of the Holy Doors. Then he says in a loud voice:
%</"d1005">
%<*"d1006">
Then the Priest places the Chalice on the Holy Table, takes the Paten from the Deacon and places it to the left of the Chalice, saying:
%</"d1006">
%<*"d1007">
Then the Priest takes the Chalice, with the Communion cloth, and says:
%</"d1007">
%<*"d1008">
Then the Priest takes the censer and censes the Offering three times, saying each time:
%</"d1008">
%<*"d1009">
Then the Priest takes the censer and censes the loaves in the form of a cross, going round the four sides of the table, the Deacon [or Server] going opposite him with the candle.
%</"d1009">
%<*"d1010">
Then the Priest thrusts the Lance into the right-hand side of the seal, beside the letters IC.NI, and says:
%</"d1010">
%<*"d1011">
Then the Priest unfolds the Antimension on the Holy Table.
%</"d1011">
%<*"d1012">
Then the Priest, and the Deacon, in his usual place, bow three times, saying:
%</"d1012">
%<*"d1013">
Then the Priest, blessing each of his vestments and kissing the cross on them, puts them on, saying:
%</"d1013">
%<*"d1014">
Then the Priest, having taken the rings on the dish, gives first to the man the gold one and says three times, as he makes the sign of the Cross with the ring on his forehead:
%</"d1014">
%<*"d1015">
Then the Priest, or Deacon, takes the censer and censes the Holy Table, the Sanctuary, the principal icons and the People, coming out a little from the holy Doors.
%</"d1015">
%<*"d1016">
Then the Priest, raising the Prosphora and the Lance level with his forehead, says:
%</"d1016">
%<*"d1017">
Then the Priest, taking a prosphora in his left hand and the Lance in his right, makes the sign of the Cross three times over the seal with the Lance, saying each time:
%</"d1017">
%<*"d1018">
Then the Priest, thrusting the lance into the right hand side of the prosphora, takes out the Lamb, saying:
%</"d1018">
%<*"d1019">
Then the Resurrection Apolytikion of the current Mode.
%</"d1019">
%<*"d1020">
Then the Resurrection Apolytikion.
%</"d1020">
%<*"d1021">
Then the Song of Symeon.
%</"d1021">
%<*"d1022">
Then the following Troparion:
%</"d1022">
%<*"d1023">
Then the following prayer:
%</"d1023">
%<*"d1024">
Then the following troparia are sung in the 4th Mode: Mode 4.
%</"d1024">
%<*"d1025">
Then the following troparia are sung in the 4th Tone:
%</"d1025">
%<*"d1026">
Then the icon of the Forerunner:
%</"d1026">
%<*"d1027">
Then the little Litany by the Deacon.
%</"d1027">
%<*"d1028">
Then they both wash their hands, saying the Psalm:
%</"d1028">
%<*"d1029">
Then they bow three times and say:
%</"d1029">
%<*"d1030">
Then they go and venerate the Icons. As they kiss the icon of Christ they say:
%</"d1030">
%<*"d1031">
Then they make three bows before the Prothesis, saying each time:
%</"d1031">
%<*"d1032">
Then, from the same or another prosphora, he cuts particles for the living, and places them on the paten below the Lamb, saying:
%</"d1032">
%<*"d1033">
Then, from the same or another prosphora, he remembers the Bishop who ordained him, if he is no longer alive, and others of the departed, whom he wishes, by name, saying for each:
%</"d1033">
%<*"d1034">
Then, having taken the silver ring, he does the same to the woman, saying:
%</"d1034">
%<*"d1035">
Then, on all days:
%</"d1035">
%<*"d1036">
Then, taking the same, or a third, prosphora, he cuts smaller particles for the nine ranks of the Saints, and places them on the paten, to the left of the Lamb, that is to the priest's right hand, in three rows or portions, so starting the first rank, saying:
%</"d1036">
%<*"d1037">
Then:
%</"d1037">
%<*"d1038">
Theotokion
%</"d1038">
%<*"d1039">
Theotokion.
%</"d1039">
%<*"d1040">
Theotokion. To the same melody.
%</"d1040">
%<*"d1041">
They bow their heads asking forgiveness of the people, and so go into the Altar, the Priest through the north door and the Deacon through the south, saying:
%</"d1041">
%<*"d1042">
They go to the table of the Prothesis, and having made three bows they kiss the covered Holy Gifts, saying:
%</"d1042">
%<*"d1043">
They make three bows before the holy Table, and kiss the holy Gospel and the holy Table.
%</"d1043">
%<*"d1044">
Things good and profitable for our souls, and peace for the world, let us ask of the Lord.
%</"d1044">
%<*"d1045">
This is the chosen and Holy Day, the first of all Sabbaths, the Sovereign and Lady, the Feast of Feasts, and the Festival of Festivals on which we bless Christ unto all ages.
%</"d1045">
%<*"d1046">
Those inside:
%</"d1046">
%<*"d1047">
Those things that are good and profitable for our souls, and peace for the world, let us ask of the Lord.
%</"d1047">
%<*"d1048">
Though you descended into the tomb, O Immortal, yet you destroyed the power of Hades; and you arose as victor, O Christ God, calling to the Myrrh-bearing women, \u2018Rejoice!' and giving peace to your Apostles, O you who grant resurrection to the fallen.
%</"d1048">
%<*"d1049">
Through the grace and compassion and love towards mankind of your only-begotten Son, with whom you are blessed, together with your all-holy, good and life-giving Spirit, now and for ever, and to the ages of ages.
%</"d1049">
%<*"d1050">
Through the prayers of our holy Fathers, Lord Jesus Christ our God, have mercy on us and save us.
%</"d1050">
%<*"d1051">
Through the prayers of our holy Fathers, Lord Jesus Christ our God, have mercy on us.
%</"d1051">
%<*"d1052">
Through the prayers of our holy fathers, Lord Jesus Christ, our God, have mercy upon us.
%</"d1052">
%<*"d1053">
Through the prayers of the Mother of God, O Merciful One, blot out the multitude of my transgressions.
%</"d1053">
%<*"d1054">
To you, O Lord.
%</"d1054">
%<*"d1055">
To you, most devout Deacon [and Monk] N is granted communion in the precious and all-holy Blood of our Lord and God and Saviour, Jesus Christ, for the forgiveness of your sins and for eternal life.
%</"d1055">
%<*"d1056">
To you, my Champion and Commander, I your City, * saved from disasters, dedicate, O Mother of God, * hymns of victory and thanksgiving; * but as you have unassailable might, * from every kind of danger now deliver me, * that I may cry to you, \u2018Hail, Bride without bridegroom!'
%</"d1056">
%<*"d1057">
Today is the crowning moment of our salvation, and the unfolding of the eternal mystery: the Son of God becomes the Son of the Virgin, and Gabriel brings the good tidings of grace. Therefore with him let us also cry aloud to the Mother of God: Hail, full of grace! The Lord is with you.
%</"d1057">
%<*"d1058">
Today is the prelude of the good pleasure of God, and the proclaiming of the salvation of mankind. In the Temple of God the Virgin is revealed, and beforehand she announces Christ to all. To her then let us cry aloud with mighty voice: Hail, the fulfilment of the Creator's dispensation!
%</"d1058">
%<*"d1059">
Today open wide your hearts, all you believers, and receive with mind made pure the Lord who comes to dwell in them, as before the feast you sing aloud his praise.
%</"d1059">
%<*"d1060">
Today salvation has come to the world. * Let us sing to him who rose from the tomb, the Author of our life. * For destroying death by death, * he has given us the victory * and his great mercy.
%</"d1060">
%<*"d1061">
Today the Lord has come to the streams of Jordan, and cries aloud to John, \u2018Do not be afraid to baptize me; for I have come to save Adam, the First-formed'.
%</"d1061">
%<*"d1062">
Today the Virgin and Mother of God, Mary, the untouched bridal chamber of the heavenly Bridegroom, is being brought to birth from a barren womb by God's counsel, to be made ready as the chariot of the Word of God; for to this she was predestined, the gate of God and Mother of true life.
%</"d1062">
%<*"d1063">
Today the Virgin comes to the cave, to give birth ineffably to the eternal Word. Hearing this, dance, O inhabited world! Glorify, with Angels and with Shepherds, him who willed to be made manifest a little Child, God before the ages.
%</"d1063">
%<*"d1064">
Today the Virgin gives birth to him who is above all being, and the earth offers the cave to him whom no one can approach. Angels with Shepherds give glory, while Magi journey with a star, for to us there has been born a little Child, God before the ages.
%</"d1064">
%<*"d1065">
Today the inhabited world celebrates Anne's conceiving, which took place through God; for she bore in her womb the one who above reason bore the Word.
%</"d1065">
%<*"d1066">
Today the whole world has been filled with joy, at the glorious festival of the Mother of God, as it cries, \u2018She is the heavenly Tabernacle'.
%</"d1066">
%<*"d1067">
Today you have appeared to the inhabited world, and your light, O Lord, has been signed upon us, who with knowledge sing your praise. You have come, you have appeared, the unapproachable Light.
%</"d1067">
%<*"d1068">
Tone 1.
%</"d1068">
%<*"d1069">
Tone 1. By Monk John.
%</"d1069">
%<*"d1070">
Tone 1. Model Melody.
%</"d1070">
%<*"d1071">
Tone 1. Model Melody. (By St Romanos)
%</"d1071">
%<*"d1072">
Tone 2.
%</"d1072">
%<*"d1073">
Tone 2. Mode 2. Kontakion of the Mother of God.
%</"d1073">
%<*"d1074">
Tone 2. Model Melody.
%</"d1074">
%<*"d1075">
Tone 2. Nor tomb nor death.
%</"d1075">
%<*"d1076">
Tone 2. Seeking things above.
%</"d1076">
%<*"d1077">
Tone 3.
%</"d1077">
%<*"d1078">
Tone 3. Model Melody. (By St Romanos)
%</"d1078">
%<*"d1079">
Tone 3. Today the Virgin.
%</"d1079">
%<*"d1080">
Tone 4.
%</"d1080">
%<*"d1081">
Tone 4. Lifted up on the Cross.
%</"d1081">
%<*"d1082">
Tone 4. Speedily anticipate.
%</"d1082">
%<*"d1083">
Tone 5
%</"d1083">
%<*"d1084">
Tone 5.
%</"d1084">
%<*"d1085">
Tone 6.
%</"d1085">
%<*"d1086">
Tone 6. Model Melody.
%</"d1086">
%<*"d1087">
Tone 6. Model Melody. (By St Romanos)
%</"d1087">
%<*"d1088">
Tone 7.
%</"d1088">
%<*"d1089">
Tone 7. Model Melody.
%</"d1089">
%<*"d1090">
Tone 8.
%</"d1090">
%<*"d1091">
Tone 8. Model Melody.
%</"d1091">
%<*"d1092">
Tone 8. Model Melody. (By St Romanos)
%</"d1092">
%<*"d1093">
Trisagion etc.
%</"d1093">
%<*"d1094">
Troparia.
%</"d1094">
%<*"d1095">
Unto you will I cry, O Lord my God.
%</"d1095">
%<*"d1096">
Upon the divine watch let the God-inspired Habakkuk stand with us and show forth the radiant Angel fervently saying: Today is salvation to the world, for Christ is risen for he is Almighty.
%</"d1096">
%<*"d1097">
Verse 1: Give thanks to the Lord, for he is good: his mercy endures for ever.
%</"d1097">
%<*"d1098">
Verse 2: All the nations surrounded me, but in the name of the Lord I drove them back.
%</"d1098">
%<*"d1099">
Verse 3: This is the Lord's doing, and it is marvellous in our eyes.
%</"d1099">
%<*"d1100">
Verse: Blessed is the one whom you have chosen and taken, O Lord.
%</"d1100">
%<*"d1101">
Verse: The Lord is the defender of my life, of whom shall I be afraid?
%</"d1101">
%<*"d1102">
Verses of Admonition.
%</"d1102">
%<*"d1103">
Virgin Mother of God, hail Mary, full of grace; the Lord is with you; blessed are you among women; and blessed is the fruit of your womb; for you gave birth to the Saviour of our souls.
%</"d1103">
%<*"d1104">
WITH THE GIFTS OF BREAD AND WINE
%</"d1104">
%<*"d1105">
Watching life's sea rising with a tempest of temptations, fleeing to your calm haven, I cry out to you: Bring my life up from corruption, O Most merciful.
%</"d1105">
%<*"d1106">
We praise you, * we bless you, * we worship you, * we glorify you, * we give you thanks * for your great glory.
%</"d1106">
%<*"d1107">
We then sing the Pascal Katavasia as follows:
%</"d1107">
%<*"d1108">
We venerate your most pure icon, loving Lord, as we ask pardon of our offences, Christ God. For by your own choice you were well-pleased to ascend the Cross in the flesh, to deliver from the slavery of the enemy those whom you had fashioned; therefore with thanksgiving we cry to you: You have filled all things with joy, our Saviour, by coming to save the world.
%</"d1108">
%<*"d1109">
We, who in a mystery represent the Cherubim and sing the thrice\u001Eholy hymn to the life\u001Egiving Trinity, let us now lay aside every care of this life.
%</"d1109">
%<*"d1110">
When all have communicated, the Priest hands the Chalice to the Deacon, who places it on the holy Table again.
%</"d1110">
%<*"d1111">
When an Artoklasia is celebrated at Matins, the blessing of loaves takes place at the very end of Matins, that is, immediately after the Doxology.
%</"d1111">
%<*"d1112">
When it is time, he enters the Church and, together with the Deacon, makes a bow to the Bishop's stall and then three bows towards the east, in front of the closed Holy Doors, saying at each one:
%</"d1112">
%<*"d1113">
When the Most High came down and confused the tongues, he parted the nations. When he divided out the tongues of fire, he called all to unity; and with one voice we glorify the All-holy Spirit.
%</"d1113">
%<*"d1114">
When the Singers reach the end of the first part of the Cherubic Hymn, the Deacon and the Priest come out from the north door of the Sanctuary, preceded by exapteryga, lights and incense, and pass through the north aisle and the middle of the Nave as they make the Great Entrance. As they process they proclaim, one after the other:
%</"d1114">
%<*"d1115">
When the moment to start the Liturgy has come the Priest and Deacon stand together before the holy Table and the Priest says:
%</"d1115">
%<*"d1116">
When the prayer is finished, the Deacon, in a low voice, says to the Priest:
%</"d1116">
%<*"d1117">
When the stone had been sealed by the Jews, and while soldiers were guarding your most pure Body, you rose, O Saviour, on the third day, giving life to the world; therefore the heavenly Powers cried out to you, Giver of life: Glory to your Resurrection, O Christ! Glory to your Kingdom! Glory to your dispensation, only lover of mankind!
%</"d1117">
%<*"d1118">
When the troparion is finished:
%</"d1118">
%<*"d1119">
When the women Disciples of the Lord had learnt from the Angel the joyful message of the Resurrection, casting away the ancestral condemnation, triumphantly they said to the Apostles: Death has been despoiled, Christ God has been raised, granting the world his great mercy.
%</"d1119">
%<*"d1120">
When there is an Artoklasia, the loaves with wheat, wine and oil are placed in the centre of the Church; and, while the Chanters sing one of the troparia of the Liti of the Saint whose feast is being celebrated (if the feast has no Liti, the Apolytikion is sung), the Priest and Deacon come from the Sanctuary by the Holy Doors, with lights and incense, and take their stand in the middle of the Church.
%</"d1120">
%<*"d1121">
When you come upon the earth, O God, in glory, and the whole universe trembles, while a river of fire flows before the seat of judgement, and books are opened and all secrets are disclosed, then deliver me from the unquenchable fire, and count me worthy to stand at your right hand, O Judge most just.
%</"d1121">
%<*"d1122">
When you had fulfilled your dispensation for us, and united things on earth with things in heaven, you were taken up in glory, Christ our God; in no way divided, but remaining inseparable, you cried to those who loved you, \u2018I am with you, and there is no one against you'.
%</"d1122">
%<*"d1123">
When you went down to death, O immortal life, then you slew Hades with the lightning flash of your Godhead; but when from the depths below the earth you raised the dead, all the Powers beyond the heavens cried out: Giver of life, Christ our God, glory to you!
%</"d1123">
%<*"d1124">
Whenever you have had Communion Of the life-giving and transcendent gifts, At once give praise and offer heartfelt thanks, And from your soul say fervently to God: Glory to you, O God. Glory to you, O God. Glory to you, O God.
%</"d1124">
%<*"d1125">
While Communion is being given the following is sung, as many times as is necessary for the number of communicants:
%</"d1125">
%<*"d1126">
While it is being sung, the Priest, in front of the Holy Table, reads, in a low voice, the
%</"d1126">
%<*"d1127">
While the Alleluia is being sung the *Deacon, taking the censer with incense, approaches the Priest, and having received a blessing for the incense he censes the book of the Gospel, the Holy Table all round, the whole sanctuary, the Priest and, coming out a little from the Holy Doors, the principal icons and the People.
%</"d1127">
%<*"d1128">
While the Doxastikon of the Beatitudes, or the Third Antiphon, is being sung, the Priest and Deacon, standing in front of the Holy Table, make three bows; then the Priest takes the holy Gospel and gives it to the Deacon, who kisses the Priest's hand. And so they come out through the north door, preceded by lights, and make the Little Entrance. Standing in the middle of the church they bow their heads.
%</"d1128">
%<*"d1129">
While the grave was sealed, Christ God, you dawned as life from the tomb; and while the doors were shut, you came, the resurrection of all, to your Disciples, through them renewing a right spirit in us, according to your great mercy.
%</"d1129">
%<*"d1130">
While this is being said the Bishop, keeping his hand on the head of the candidate, prays as follows in a low voice:
%</"d1130">
%<*"d1131">
While this is being sung the Deacon bows, leaves his place and goes and stands in front of the icon of the Mother of God, looking towards the icon of Christ, holding his orarion in the three fingers of his right hand.
%</"d1131">
%<*"d1132">
While this petition is being said, the Chanters sing
%</"d1132">
%<*"d1133">
Who is this king of glory?
%</"d1133">
%<*"d1134">
Who will declare his generation?
%</"d1134">
%<*"d1135">
Whoever wishes to be great among you, shall be your servant.
%</"d1135">
%<*"d1136">
Wisdom
%</"d1136">
%<*"d1137">
Wisdom Arise, let us hear the Holy Gospel.
%</"d1137">
%<*"d1138">
Wisdom let us attend
%</"d1138">
%<*"d1139">
Wisdom let us attend.
%</"d1139">
%<*"d1140">
Wisdom!
%</"d1140">
%<*"d1141">
Wisdom, stand upright! Let us listen to the Holy Gospel.
%</"d1141">
%<*"d1142">
Wisdom, stand upright. Let us listen to the holy Gospel.
%</"d1142">
%<*"d1143">
Wisdom.
%</"d1143">
%<*"d1144">
Wisdom. Stand steadfast. Let us hear the Holy Gospel.
%</"d1144">
%<*"d1145">
Wisdom. Stand upright.
%</"d1145">
%<*"d1146">
Wisdom. Stand upright. Let us listen to the Holy Gospel. Peace to all.
%</"d1146">
%<*"d1147">
Wisdom. Stand upright. Let us listen to the holy Gospel.
%</"d1147">
%<*"d1148">
Wisdom; stand and attend; let us hear the Holy Gospel; peace be with all.
%</"d1148">
%<*"d1149">
With his meddling right hand Thomas explored your life- giving side, Christ God; for, the doors being shut when you entered, he cried out to you with the rest of the Apostles, \u2018You are my Lord and my God'.
%</"d1149">
%<*"d1150">
With the Saints give rest, O Christ, to the souls of your servants where there is neither toil nor grief nor sighing, but life everlasting.
%</"d1150">
%<*"d1151">
With the Spirit you have sanctified your Church on earth, O Christ, by anointing her with the oil of your gladness.
%</"d1151">
%<*"d1152">
With the spirits of the righteous made perfect in death give rest, O Saviour, to the soul of your servant; keeping it for the life of blessedness with you, O Lover of mankind.
%</"d1152">
%<*"d1153">
With the spirits of the righteous made perfect in death give rest, O Saviour, to the soul(s) of your servant(s); keeping it (them) for the life of blessedness with you, O Lover of mankind.
%</"d1153">
%<*"d1154">
With the streams of your tears you cultivated the barren desert, and with deep sighings from the heart you made your toils bring forth fruit a hundredfold, and you became a beacon, shining in all the world by your wonders, our venerable Father John; intercede with Christ God that our souls may be saved.
%</"d1154">
%<*"d1155">
With your body, O Christ, you were in the tomb, with your soul in Hades as God, in Paradise with the Thief, on the throne with Father and the Spirit, filling all things, yet yourself uncircumscribed.
%</"d1155">
%<*"d1156">
Woman: Amen.
%</"d1156">
%<*"d1157">
You Holy Martyrs, who proclaimed the Lamb of God, and like lambs were slain, and have been taken over to the unending life which knows no ageing, plead with him to grant us abolition of our debts.
%</"d1157">
%<*"d1158">
You abolished death by your Cross, you opened Paradise to the Thief, you transformed the Myrrhbearers' lament, and ordered your Apostles to proclaim that you had risen, O Christ God, granting the world your great mercy.
%</"d1158">
%<*"d1159">
You are glorified above all, Christ, our God, who established our Fathers as beacons on the earth, and through them guided us all to the true faith. Greatly compassionate Lord, glory to you!
%</"d1159">
%<*"d1160">
You are our God who descended into Hades and did away with the pains of those who had been bound; give rest, O Saviour, also to the soul(s) of your servant(s).
%</"d1160">
%<*"d1161">
You are our God who descended into Hell and did away with the pains of those who had been bound; give rest, O Saviour, also to the soul of your servant.
%</"d1161">
%<*"d1162">
You came down from above, O Compassionate, you accepted burial for three days, that you might free us from the passions. Our life and resurrection, Lord, glory to you!
%</"d1162">
%<*"d1163">
You have smitten me with longing, O Christ, and changed me by your divine love; but with your immaterial fire burn up my sins and count me worthy to be filled with delight in you, that as I leap for joy, O Good One, I may magnify your first and second Comings.
%</"d1163">
%<*"d1164">
You justified the Forefathers by faith, through them betrothing in advance the Church from the nations. Let the Saints exult in glory, for from their seed there is a glorious fruit, she who bore you without seed. At their intercessions, O Christ God, save our souls.
%</"d1164">
%<*"d1165">
You received divine grace from heaven, and through your lips you teach us all to worship one God in Trinity, venerable John Chrysostom, wholly blessed. Fittingly we praise you, for you are a teacher who makes clear things divine.
%</"d1165">
%<*"d1166">
You redeemed us from the curse of the law by your precious blood; nailed to the Cross and pierced by the lance, you became a source of immortality for all. Our Saviour, glory to you.
%</"d1166">
%<*"d1167">
You sanctified a virgin womb by your birth, and fittingly blessed Symeon's hands; you have come now too and saved us, O Christ God. But give peace to your commonwealth in times of war, and strengthen its Rulers, whose friend you are, for you alone love mankind.
%</"d1167">
%<*"d1168">
You see the deifying Blood: have fear. It is a coal consuming the unworthy. God's Body deifies and nourishes me strangely; both my spirit and my mind.
%</"d1168">
%<*"d1169">
You were taken up in glory, Christ our God, giving joy to your Disciples by the promise of the Holy Spirit, when through the blessing they had been assured that you are the Son of God, the Redeemer of the world.
%</"d1169">
%<*"d1170">
You were transfigured on the mountain, O Christ God, showing your Disciples your glory, as far as they could bear it. At the prayers of the Mother of God make your everlasting light shine also on us sinners. Giver of light, glory to you.
%</"d1170">
%<*"d1171">
You were transfigured on the mountain, and your Disciples beheld your glory, O Christ God, as far as they were able; that when they saw you crucified they might know that your suffering was voluntary, and might proclaim to the world that you are truly the brightness of the Father.
%</"d1171">
%<*"d1172">
You who willingly give me your flesh for food, Who are a fire consuming the unworthy; Do not burn me up, my Maker; But penetrate the structure of my limbs, All my joints, my inner parts, my heart. Burn up the thorns of all my offences, Purify my soul and sanctify my mind. Strengthen my knees, together with my bones; Enlighten the five-fold simpleness of my senses; Nail down the whole of me with fear of you. Always protect, guard and keep me From every soul-destroying deed and word. Hallow me, purify me, bring me to harmony, And give me beauty, understanding, light; Show me to be your dwelling, the Spirit's house alone, And no more the dwelling place of sin; That, by the entrance of Communion, Every evil-doer, every passion May flee from me, your house, as from a fire. As intercessors I bring you all the Saints, The companies of the Bodiless Hosts, Your Forerunner, the wise Apostles, And with them your most pure and holy Mother; Accept their prayers, O my compassionate Christ, And make your servant a child of light. For you alone are the hallowing of our souls, O Good One, and their brightness, And fittingly to you, as to our God and Master, We all give praise and glory every day.
%</"d1172">
%<*"d1173">
Your Nativity, O Christ, our God, has made the light of knowledge dawn on the world; for through it those who adored the stars were taught by a star to worship you, the Sun of righteousness, and to know you, the Dawn from above. Lord, glory to you.
%</"d1173">
%<*"d1174">
Your good Spirit will guide me in an upright land.
%</"d1174">
%<*"d1175">
Your hands made me and fashioned me; make me understand, and I shall learn your commandments. Have mercy on your servant.
%</"d1175">
%<*"d1176">
Your honoured cincture, Mother of God, which encircled your womb that received God, is for your City invincible might and an inexhaustible treasury of blessings, ever-virgin who alone gives birth.
%</"d1176">
%<*"d1177">
Your nativity, O Mother of God, brought joy to the whole inhabited world, for from you there dawned the Sun of righteousness, Christ our God. He abolished the curse and gave the blessing; and by making death of no effect he bestowed on us eternal life.
%</"d1177">
%<*"d1178">
Your right hand, Lord, has been glorified in strength; your right hand, Lord, has shattered enemies, and by the greatness of your glory you have crushed the adversaries.
%</"d1178">
%<*"d1179">
Your virtue, O Christ, has covered the heavens, and the earth is full of your praise.
%</"d1179">
%<*"d1180">
[A] Blessed are you, O Lord, teach me your statutes.
%</"d1180">
%<*"d1181">
[A] For he spoke and they came into being; he commanded and they were created.
%</"d1181">
%<*"d1182">
[A] His holy ones will exult in glory, and rejoice upon their beds.
%</"d1182">
%<*"d1183">
[A] His praise is above earth and heaven, and he will exalt the horn of his people.
%</"d1183">
%<*"d1184">
[A] Kings of the earth and all peoples, rulers and all judges of the earth;
%</"d1184">
%<*"d1185">
[A] Let them praise his name in the dance; let them sing his praise with timbrel and with harp.
%</"d1185">
%<*"d1186">
[A] Mountains and all hills, fruiting trees and all cedars;
%</"d1186">
%<*"d1187">
[A] Praise him for his mighty acts; praise him according to the greatness of his majesty.
%</"d1187">
%<*"d1188">
[A] Praise him with timbrel and dance; praise him with strings and pipe.
%</"d1188">
%<*"d1189">
[A] Praise him, sun and moon; praise him, all you stars and light.
%</"d1189">
%<*"d1190">
[A] Praise the Lord from the earth; praise him, you sea-monsters and all deeps;
%</"d1190">
%<*"d1191">
[A] Praise the Lord from the heavens; praise him in the highest. To you praise is due, O God.
%</"d1191">
%<*"d1192">
[A] Sing to the Lord a new song, his praise in the Church of the holy ones.
%</"d1192">
%<*"d1193">
[A] To exact vengeance among the nations, punishments among the peoples.
%</"d1193">
%<*"d1194">
[A] To execute upon them the judgement that is decreed; such glory will be for all his holy ones.
%</"d1194">
%<*"d1195">
[B] A hymn for all his holy ones; for the children of Israel, a people that draws near him.
%</"d1195">
%<*"d1196">
[B] Beasts of the wild, and all cattle, creeping things and winged birds;
%</"d1196">
%<*"d1197">
[B] Fire and hail, snow and ice and storm-wind; things that do his word.
%</"d1197">
%<*"d1198">
[B] For the Lord is well-pleased with his people; he will exalt the meek with salvation.
%</"d1198">
%<*"d1199">
[B] He established them for ever and ever; he made an ordinance, and it shall not pass away.
%</"d1199">
%<*"d1200">
[B] Let Israel rejoice in him that made him, let the children of Sion be joyful in their king.
%</"d1200">
%<*"d1201">
[B] Praise God in his saints; praise him in the firmament of his power.
%</"d1201">
%<*"d1202">
[B] Praise him with the sound of the trumpet; praise him with lute and harp.
%</"d1202">
%<*"d1203">
[B] Praise him with tuneful cymbals; praise him with loud cymbals. Let everything that has breath praise the Lord.
%</"d1203">
%<*"d1204">
[B] Praise him, all his angels: Praise him, all his Powers. To you praise is due, O God.
%</"d1204">
%<*"d1205">
[B] Praise him, you highest heavens and you waters that are above the heavens. Let them praise the name of the Lord.
%</"d1205">
%<*"d1206">
[B] The high praises of God in their mouths, and two-edged swords in their hands,
%</"d1206">
%<*"d1207">
[B] To bind their kings in fetters; and their nobles in shackles of iron.
%</"d1207">
%<*"d1208">
[B] Young men and maidens: old men and youths together, let them praise the name of the Lord; for his name alone has been exalted.
%</"d1208">
%<*"d1209">
wisdom
%</"d1209">
%<*"d1210">
\u00A0Amen.
%</"d1210">
%<*"d1211">
\u00A0Glory be to Thee, O our God, glory be to Thee.
%</"d1211">
%<*"d1212">
\uFEFF
%</"d1212">
%<*"d1213">
Catechumen
%</"d1213">
%<*"d1214.title.doc">
The Apostle
%</"d1214.title.doc">
%<*"d1214.title.toc">
The Apostle
%</"d1214.title.toc">
%<*"d1214.title.heading">
The Apostle
%</"d1214.title.heading">
%<*"d1215.title.doc">
The Gospel
%</"d1215.title.doc">
%<*"d1215.title.toc">
The Gospel
%</"d1215.title.toc">
%<*"d1215.title.heading">
The Gospel
%</"d1215.title.heading">
%<*"d1216">
And so after each petition.
%</"d1216">
%<*"d1217">
For our Archbishop
%</"d1217">
%<*"d1218">
N.
%</"d1218">
%<*"d1219">
, for the honoured order of presbyters, for the diaconate in Christ, for all the clergy and the people, let us pray to the Lord.
%</"d1219">
%<*"d1220a">
The servant of God
%</"d1220a">
%<*"d1220b">
N.
%</"d1220b">
%<*"d1220c">
betroths himself to the servant of God
%</"d1220c">
%<*"d1220d">
M.
%</"d1220d">
%<*"d1220e">
,in the name of the Father and of the Son and of the Holy Spirit. Amen.
%</"d1220e">
%<*"d3t">
Three Times
%</"d3t">
%<*"d03t">
three times
%</"d03t">
%<*"d1221a">
The servant of God 
%</"d1221a">
%<*"d1221b">
betroths himself to the servant of God 
%</"d1221b">
%<*"d1221c">
, in the name of the Father and of the Son and of the Holy Spirit. Amen.
%</"d1221c">
%<*"d)">
)
%</"d)">